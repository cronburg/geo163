%-----------------------------------------------------------------------------
%
%               Template for sigplanconf LaTeX Class
%
% Name:         sigplanconf-template.tex
%
% Purpose:      A template for sigplanconf.cls, which is a LaTeX 2e class
%               file for SIGPLAN conference proceedings.
%
% Guide:        Refer to "Author's Guide to the ACM SIGPLAN Class,"
%               sigplanconf-guide.pdf
%
% Author:       Paul C. Anagnostopoulos
%               Windfall Software
%               978 371-2316
%               paul@windfall.com
%
% Created:      15 February 2005
%
%-----------------------------------------------------------------------------


\documentclass[10pt,preprint]{sigplanconf}

% The following \documentclass options may be useful:

% preprint      Remove this option only once the paper is in final form.
% 10pt          To set in 10-point type instead of 9-point.
% 11pt          To set in 11-point type instead of 9-point.
% authoryear    To obtain author/year citation style instead of numeric.

\usepackage{fancyvrb}
\usepackage{amsmath}
\usepackage{minted}
\usepackage{hyperref}

% pstricks?:
\usepackage{graphicx}
\usepackage{mathtools}
\usepackage[pdf]{pstricks}
\usepackage{array}
\usepackage{amsmath, amssymb, graphics, setspace}
\newcommand{\mathsym}[1]{{}}
\newcommand{\unicode}[1]{{}}

% \cc{my_c_code}
\newmintinline[cc]{c}{fontsize=\small}
\newmintinline[java]{java}{fontsize=\small}
\newmintinline[py]{python}{fontsize=\small}

\begin{document}

\special{papersize=8.5in,11in}
\setlength{\pdfpageheight}{\paperheight}
\setlength{\pdfpagewidth}{\paperwidth}

% Re-add if accepted:
\conferenceinfo{CONF 'yy}{Month d--d, 20yy, City, ST, Country} 
\copyrightyear{20yy} 
\copyrightdata{978-1-nnnn-nnnn-n/yy/mm} 
\doi{nnnnnnn.nnnnnnn}

% Uncomment one of the following two, if you are not going for the 
% traditional copyright transfer agreement.

%\exclusivelicense                % ACM gets exclusive license to publish, 
                                  % you retain copyright

%\permissiontopublish             % ACM gets nonexclusive license to publish
                                  % (paid open-access papers, 
                                  % short abstracts)

%\titlebanner{banner above paper title}        % These are ignored unless
%\preprintfooter{short description of paper}   % 'preprint' option specified.

\title{Dynamic Visibility of a Point in a Polygon}
%\subtitle{Subtitle Text, if any}

% For empty space pre-submission:
%\authorinfo{\vspace{-50ex}}{\vspace{-50ex}}{\vspace{-50ex}}

\authorinfo{Karl Cronburg}
           {Tufts University}
           {karl@cs.tufts.edu}
%\authorinfo{Samuel Z. Guyer}
%           {Tufts University}
%           {sguyer@cs.tufts.edu}

%\authorinfo{Name2\and Name3}
%           {Affiliation2/3}
%           {Email2/3}

\maketitle

\begin{abstract}
In this work we present a time-optimal algorithm for maintaining
the visibility polygon of a point in a polygon in the presence of
small point movements.
The work of \cite{dynamic-visibility} maintains a space complexity
of $O(n)$ ($n$ input polygon vertices) and describes their data
structures in the dual. In contrast we spare no expense with space,
and instead look at what inherent time complexity limitations arise
from the dynamic visibility problem.
\end{abstract}

% Re-add if accepted:
%\category{CR-number}{subcategory}{third-level}

% general terms are not compulsory anymore, 
% you may leave them out
%\terms
%term1, term2

\keywords
Visibility, polygon, geometry

%\section{Introduction}

\section{Introduction} \label{sec:intro}

Given effectively unlimited precomputation time, and space we show
that the visibility of a point inside a simple polygon reduces to
(at runtime) the problem of detecting segment intersection with a
much smaller convex polygon. These convex polygons can be further
triangulated to 

Our algorithm is directly applicable to the implementation
of games and simulations involving the continuous motion of points
through a closed and finite partition of a $2$-dimensional space.
For example one might have a particle simulation involving at least
$p \in \Omega(n^2)$ particles ($n$ polygon vertices). In this situation,
the naiive worst case space complexity requires computing $O(n^2)$
distinct visibility polygons each in $O(n)$ time without any
precomputation or memoization optimizations.

Furthermore, many dynamic visibility applications realistically have
the self-imposed limitation of small point velocities. Relatively
low velocities lends itself to lowering the cost of computing the
next segment intersection to cause a \emph{change} in the visibility
polygon. By \emph{change} we mean:

\begin{itemize}
\item A visibility \emph{window} merges with an edge in the polygon
(Figure~\ref{fig:window-edge-merge}).
\item A visibility \emph{window} passes over a vertex in the polygon,
changing which edge it lands on (Figure~\ref{fig:window-vertex-pass}).
\item A \emph{window} is otherwise created or removed / an edge becomes
visibile or invisible (Figure~\ref{fig:edge-change}).
\end{itemize}

All of these cases are defined by our point crossing over an infinite ray
extending from an edge on a reflex vertex, pointing into the polygon
(dashed lines in Figure \ref{fig:visibility-regions}). As we know from the algorithm
for computing the kernel of a polygon, these rays intersect with each
other to form a partitioning of the closed space inside our polygon
into convex subpolygons. In the \emph{Algorithm} section we discuss how
further decomposing these convex subpolygons into Delaunay triangulations
allows us to detect visibility changes in constant time.
% Number of partitions is $\omega(1)$ and $O(n^2)$ convex subpolygons.



\section{Algorithm \& Data Structures} \label{sec:algorithm}

The goal of this algorithm is to have average case constant time
visibility polygon updates in the presence of a moving visibility point. This
\py{UpdateVisibility} pseudocode gives a top-level view of what the
algorithm does:

\begin{minted}{python}
def UpdateVisibility(point): # O(1) on average
  move = Edge(point, point.nextPosition):
  # O(1) neighbors on average
  for neighbor in point.currDelaunay.neighbors:
    if Intersect(move, neighbor):
      return neighbor
  return point.currDelaunay
\end{minted}

This assumes we have a graph of Delaunay triangles, each of which have
associated with them $O(1)$ instructions for how to compute each of the
windows in their visibility polygon given the location of a point inside
the DT.

\subsection{UpdateVisibility Details}

In \py{UpdateVisibility} we are finding the Delaunay triangle (DT) that our
point is transitioning into by checking which neighboring DT
intersects with the point's segment of movement.

When the point moves from one DT to another we have one of two cases:

\begin{enumerate}
  \item The new DT is inside the same convex \emph{visibility region}
    (see Figure~\ref{fig:visibility-regions}). In this
    case we need only recompute where the window points lay on the same
    edge they were on before. This can be done in $O(w)$ for $w$ window
    edges in the visibility region, which is necessarily $O(w) \in O(v)$
    for $v$ visibility polygon vertices. This is optimal because we must
    at least read off the $v$ vertices in the visibility solution. If on
    the other hand we did not need the visibility polygon for a particular
    time step, the runtime is $O(1)$ to move the point in our Delaunay
    graph.
  \item The new DT crosses a convex visibility region edge into a new
    region. Given general position, this transition requires the
    updating of exactly one edge / window in the visibility polygon
    which can be done online in $O(1)$. Again as in case $1$, we also
    recalculate where the windows land on their respective polygon
    edges in $O(w)$ time.
\end{enumerate}

\begin{figure}
  \begin{center}
    %LaTeX with PSTricks extensions
%%Creator: inkscape 0.91
%%Please note this file requires PSTricks extensions
\psset{xunit=1.0pt,yunit=1.0pt,runit=.5pt}
\begin{pspicture}(216.14173228,159.4488189)
{
\newrgbcolor{curcolor}{0 0 0}
\pscustom[linewidth=0.80000054,linecolor=curcolor,linestyle=dashed,dash=2.4 2.4]
{
\newpath
\moveto(75.17854281,75.21824865)
\lineto(74.04208213,105.14351006)
\lineto(72.77948138,141.25649589)
}
}
{
\newrgbcolor{curcolor}{0 0 0}
\pscustom[linewidth=1.00000066,linecolor=curcolor]
{
\newpath
\moveto(15.35811015,75.35683875)
\lineto(74.04208213,105.14351006)
\lineto(37.17150316,142.51918679)
\lineto(146.26804718,138.4785739)
\lineto(199.30108879,80.39476235)
\lineto(94.75022448,16.25007647)
\closepath
}
}
{
\newrgbcolor{curcolor}{0.60000002 0.60000002 0.60000002}
\pscustom[linestyle=none,fillstyle=solid,fillcolor=curcolor]
{
\newpath
\moveto(20.43717887,75.35681387)
\curveto(20.43717887,72.56324518)(18.18036254,70.29427041)(15.38632448,70.27874837)
\curveto(12.59228643,70.26322633)(10.31039078,72.50698667)(10.27934148,75.30038287)
\curveto(10.24829218,78.09377907)(12.4797505,80.38768806)(15.27344352,80.43425227)
\curveto(18.06713653,80.48081647)(20.3738297,78.26254765)(20.43592446,75.4696689)
\lineto(15.35810337,75.35681387)
\closepath
}
}
{
\newrgbcolor{curcolor}{0.15686275 0.04313726 0.04313726}
\pscustom[linewidth=0.12953625,linecolor=curcolor,linestyle=dashed,dash=0.38860851 0.38860851]
{
\newpath
\moveto(20.43717887,75.35681387)
\curveto(20.43717887,72.56324518)(18.18036254,70.29427041)(15.38632448,70.27874837)
\curveto(12.59228643,70.26322633)(10.31039078,72.50698667)(10.27934148,75.30038287)
\curveto(10.24829218,78.09377907)(12.4797505,80.38768806)(15.27344352,80.43425227)
\curveto(18.06713653,80.48081647)(20.3738297,78.26254765)(20.43592446,75.4696689)
\lineto(15.35810337,75.35681387)
\closepath
}
}
{
\newrgbcolor{curcolor}{0.60000002 0.60000002 0.60000002}
\pscustom[linestyle=none,fillstyle=solid,fillcolor=curcolor]
{
\newpath
\moveto(79.12116523,105.1435173)
\curveto(79.12116523,102.34994861)(76.86434889,100.08097384)(74.07031084,100.0654518)
\curveto(71.27627278,100.04992976)(68.99437713,102.2936901)(68.96332784,105.0870863)
\curveto(68.93227854,107.8804825)(71.16373686,110.1743915)(73.95742987,110.2209557)
\curveto(76.75112289,110.26751991)(79.05781606,108.04925108)(79.11991082,105.25637233)
\lineto(74.04208973,105.1435173)
\closepath
}
}
{
\newrgbcolor{curcolor}{0.15686275 0.04313726 0.04313726}
\pscustom[linewidth=0.12953625,linecolor=curcolor,linestyle=dashed,dash=0.38860851 0.38860851]
{
\newpath
\moveto(79.12116523,105.1435173)
\curveto(79.12116523,102.34994861)(76.86434889,100.08097384)(74.07031084,100.0654518)
\curveto(71.27627278,100.04992976)(68.99437713,102.2936901)(68.96332784,105.0870863)
\curveto(68.93227854,107.8804825)(71.16373686,110.1743915)(73.95742987,110.2209557)
\curveto(76.75112289,110.26751991)(79.05781606,108.04925108)(79.11991082,105.25637233)
\lineto(74.04208973,105.1435173)
\closepath
}
}
{
\newrgbcolor{curcolor}{0.60000002 0.60000002 0.60000002}
\pscustom[linestyle=none,fillstyle=solid,fillcolor=curcolor]
{
\newpath
\moveto(42.25057593,142.51921542)
\curveto(42.25057593,139.72564673)(39.99375959,137.45667196)(37.19972154,137.44114992)
\curveto(34.40568348,137.42562788)(32.12378783,139.66938822)(32.09273854,142.46278442)
\curveto(32.06168924,145.25618062)(34.29314756,147.55008962)(37.08684058,147.59665382)
\curveto(39.88053359,147.64321803)(42.18722676,145.4249492)(42.24932152,142.63207045)
\lineto(37.17150043,142.51921542)
\closepath
}
}
{
\newrgbcolor{curcolor}{0.15686275 0.04313726 0.04313726}
\pscustom[linewidth=0.12953625,linecolor=curcolor,linestyle=dashed,dash=0.38860851 0.38860851]
{
\newpath
\moveto(42.25057593,142.51921542)
\curveto(42.25057593,139.72564673)(39.99375959,137.45667196)(37.19972154,137.44114992)
\curveto(34.40568348,137.42562788)(32.12378783,139.66938822)(32.09273854,142.46278442)
\curveto(32.06168924,145.25618062)(34.29314756,147.55008962)(37.08684058,147.59665382)
\curveto(39.88053359,147.64321803)(42.18722676,145.4249492)(42.24932152,142.63207045)
\lineto(37.17150043,142.51921542)
\closepath
}
}
{
\newrgbcolor{curcolor}{0.60000002 0.60000002 0.60000002}
\pscustom[linestyle=none,fillstyle=solid,fillcolor=curcolor]
{
\newpath
\moveto(151.34713754,138.47856312)
\curveto(151.34713754,135.68499443)(149.0903212,133.41601966)(146.29628315,133.40049762)
\curveto(143.50224509,133.38497558)(141.22034944,135.62873592)(141.18930015,138.42213212)
\curveto(141.15825085,141.21552832)(143.38970917,143.50943731)(146.18340218,143.55600152)
\curveto(148.9770952,143.60256572)(151.28378837,141.3842969)(151.34588313,138.59141815)
\lineto(146.26806204,138.47856312)
\closepath
}
}
{
\newrgbcolor{curcolor}{0.15686275 0.04313726 0.04313726}
\pscustom[linewidth=0.12953625,linecolor=curcolor,linestyle=dashed,dash=0.38860851 0.38860851]
{
\newpath
\moveto(151.34713754,138.47856312)
\curveto(151.34713754,135.68499443)(149.0903212,133.41601966)(146.29628315,133.40049762)
\curveto(143.50224509,133.38497558)(141.22034944,135.62873592)(141.18930015,138.42213212)
\curveto(141.15825085,141.21552832)(143.38970917,143.50943731)(146.18340218,143.55600152)
\curveto(148.9770952,143.60256572)(151.28378837,141.3842969)(151.34588313,138.59141815)
\lineto(146.26806204,138.47856312)
\closepath
}
}
{
\newrgbcolor{curcolor}{0.60000002 0.60000002 0.60000002}
\pscustom[linestyle=none,fillstyle=solid,fillcolor=curcolor]
{
\newpath
\moveto(204.38020443,80.39478134)
\curveto(204.38020443,77.60121265)(202.12338809,75.33223788)(199.32935004,75.31671584)
\curveto(196.53531198,75.3011938)(194.25341633,77.54495414)(194.22236703,80.33835034)
\curveto(194.19131774,83.13174654)(196.42277606,85.42565553)(199.21646907,85.47221974)
\curveto(202.01016209,85.51878394)(204.31685526,83.30051512)(204.37895002,80.50763637)
\lineto(199.30112892,80.39478134)
\closepath
}
}
{
\newrgbcolor{curcolor}{0.15686275 0.04313726 0.04313726}
\pscustom[linewidth=0.12953625,linecolor=curcolor,linestyle=dashed,dash=0.38860851 0.38860851]
{
\newpath
\moveto(204.38020443,80.39478134)
\curveto(204.38020443,77.60121265)(202.12338809,75.33223788)(199.32935004,75.31671584)
\curveto(196.53531198,75.3011938)(194.25341633,77.54495414)(194.22236703,80.33835034)
\curveto(194.19131774,83.13174654)(196.42277606,85.42565553)(199.21646907,85.47221974)
\curveto(202.01016209,85.51878394)(204.31685526,83.30051512)(204.37895002,80.50763637)
\lineto(199.30112892,80.39478134)
\closepath
}
}
{
\newrgbcolor{curcolor}{0.60000002 0.60000002 0.60000002}
\pscustom[linestyle=none,fillstyle=solid,fillcolor=curcolor]
{
\newpath
\moveto(99.82930745,16.24996007)
\curveto(99.82930745,13.45639138)(97.57249112,11.18741661)(94.77845306,11.17189457)
\curveto(91.98441501,11.15637253)(89.70251936,13.40013287)(89.67147006,16.19352907)
\curveto(89.64042076,18.98692527)(91.87187909,21.28083426)(94.6655721,21.32739847)
\curveto(97.45926511,21.37396267)(99.76595828,19.15569385)(99.82805304,16.3628151)
\lineto(94.75023195,16.24996007)
\closepath
}
}
{
\newrgbcolor{curcolor}{0.15686275 0.04313726 0.04313726}
\pscustom[linewidth=0.12953625,linecolor=curcolor,linestyle=dashed,dash=0.38860851 0.38860851]
{
\newpath
\moveto(99.82930745,16.24996007)
\curveto(99.82930745,13.45639138)(97.57249112,11.18741661)(94.77845306,11.17189457)
\curveto(91.98441501,11.15637253)(89.70251936,13.40013287)(89.67147006,16.19352907)
\curveto(89.64042076,18.98692527)(91.87187909,21.28083426)(94.6655721,21.32739847)
\curveto(97.45926511,21.37396267)(99.76595828,19.15569385)(99.82805304,16.3628151)
\lineto(94.75023195,16.24996007)
\closepath
}
}
{
\newrgbcolor{curcolor}{0 0 0}
\pscustom[linestyle=none,fillstyle=solid,fillcolor=curcolor]
{
\newpath
\moveto(77.63940599,75.21826396)
\curveto(77.63940599,73.86475572)(76.54595886,72.76541778)(75.19222313,72.75789722)
\curveto(73.8384874,72.75037666)(72.73288912,73.83749799)(72.71784547,75.19092266)
\curveto(72.70280181,76.54434732)(73.78396277,77.65576611)(75.13753132,77.67832687)
\curveto(76.49109987,77.70088763)(77.60871277,76.62611716)(77.63879822,75.2729432)
\lineto(75.17854975,75.21826396)
\closepath
}
}
{
\newrgbcolor{curcolor}{0.15686275 0.04313726 0.04313726}
\pscustom[linewidth=0.06276144,linecolor=curcolor,linestyle=dashed,dash=0.18828419 0.18828419]
{
\newpath
\moveto(77.63940599,75.21826396)
\curveto(77.63940599,73.86475572)(76.54595886,72.76541778)(75.19222313,72.75789722)
\curveto(73.8384874,72.75037666)(72.73288912,73.83749799)(72.71784547,75.19092266)
\curveto(72.70280181,76.54434732)(73.78396277,77.65576611)(75.13753132,77.67832687)
\curveto(76.49109987,77.70088763)(77.60871277,76.62611716)(77.63879822,75.2729432)
\lineto(75.17854975,75.21826396)
\closepath
}
}
{
\newrgbcolor{curcolor}{0 0 0}
\pscustom[linestyle=none,fillstyle=solid,fillcolor=curcolor]
{
\newpath
\moveto(75.17854281,75.21824865)
\lineto(103.58915174,106.91129132)
}
}
{
\newrgbcolor{curcolor}{0 0 0}
\pscustom[linewidth=1.00000066,linecolor=curcolor]
{
\newpath
\moveto(75.17854281,75.21824865)
\lineto(103.58915174,106.91129132)
}
}
{
\newrgbcolor{curcolor}{0 0 0}
\pscustom[linestyle=none,fillstyle=solid,fillcolor=curcolor]
{
\newpath
\moveto(93.14939136,101.91277648)
\lineto(104.46248187,107.91191118)
\lineto(99.73103098,96.01279192)
\curveto(99.06975574,99.1841622)(96.40414849,101.55883646)(93.14939136,101.91277648)
\closepath
}
}
{
\newrgbcolor{curcolor}{0 0 0}
\pscustom[linewidth=0.68750045,linecolor=curcolor]
{
\newpath
\moveto(93.14939136,101.91277648)
\lineto(104.46248187,107.91191118)
\lineto(99.73103098,96.01279192)
\curveto(99.06975574,99.1841622)(96.40414849,101.55883646)(93.14939136,101.91277648)
\closepath
}
}
{
\newrgbcolor{curcolor}{0 0 0}
\pscustom[linestyle=none,fillstyle=solid,fillcolor=curcolor]
{
\newpath
\moveto(108.71091194,110.2675444)
\curveto(108.71091194,108.91403616)(107.6174648,107.81469822)(106.26372907,107.80717766)
\curveto(104.90999335,107.7996571)(103.80439507,108.88677843)(103.78935141,110.2402031)
\curveto(103.77430776,111.59362776)(104.85546871,112.70504655)(106.20903727,112.72760731)
\curveto(107.56260582,112.75016807)(108.68021871,111.6753976)(108.71030416,110.32222365)
\lineto(106.2500557,110.2675444)
\closepath
}
}
{
\newrgbcolor{curcolor}{0.15686275 0.04313726 0.04313726}
\pscustom[linewidth=0.06276144,linecolor=curcolor,linestyle=dashed,dash=0.18828419 0.18828419]
{
\newpath
\moveto(108.71091194,110.2675444)
\curveto(108.71091194,108.91403616)(107.6174648,107.81469822)(106.26372907,107.80717766)
\curveto(104.90999335,107.7996571)(103.80439507,108.88677843)(103.78935141,110.2402031)
\curveto(103.77430776,111.59362776)(104.85546871,112.70504655)(106.20903727,112.72760731)
\curveto(107.56260582,112.75016807)(108.68021871,111.6753976)(108.71030416,110.32222365)
\lineto(106.2500557,110.2675444)
\closepath
}
}
{
\newrgbcolor{curcolor}{0 0 0}
\pscustom[linestyle=none,fillstyle=solid,fillcolor=curcolor]
{
\newpath
\moveto(74.52597633,105.42291118)
\curveto(73.81502773,105.42291118)(73.32246469,105.34161436)(73.0482872,105.17902071)
\curveto(72.77410972,105.01642707)(72.63702098,104.73906144)(72.63702098,104.34692383)
\curveto(72.63702098,104.03448898)(72.7390405,103.78581635)(72.94307956,103.60090593)
\curveto(73.15030673,103.41918362)(73.43086044,103.32832246)(73.78474068,103.32832246)
\curveto(74.27252156,103.32832246)(74.66306507,103.50048044)(74.95637122,103.84479639)
\curveto(75.25286548,104.19230046)(75.40111261,104.65298245)(75.40111261,105.22684237)
\lineto(75.40111261,105.42291118)
\lineto(74.52597633,105.42291118)
\closepath
\moveto(76.28103105,105.7863558)
\lineto(76.28103105,102.73055171)
\lineto(75.40111261,102.73055171)
\lineto(75.40111261,103.54351993)
\curveto(75.20026166,103.21833264)(74.949995,102.97763029)(74.65031263,102.82141286)
\curveto(74.35063026,102.66838355)(73.98399758,102.59186889)(73.55041458,102.59186889)
\curveto(73.0020596,102.59186889)(72.56528849,102.74489821)(72.24010124,103.05095683)
\curveto(71.9181021,103.36020357)(71.75710253,103.7730639)(71.75710253,104.28953783)
\curveto(71.75710253,104.89209075)(71.95795348,105.34639653)(72.35965538,105.65245515)
\curveto(72.76454539,105.95851378)(73.36709823,106.11154309)(74.16731392,106.11154309)
\lineto(75.40111261,106.11154309)
\lineto(75.40111261,106.19762208)
\curveto(75.40111261,106.60251214)(75.26721198,106.91494698)(74.99941071,107.13492662)
\curveto(74.73479756,107.35809437)(74.36178865,107.46967824)(73.88038399,107.46967824)
\curveto(73.5743254,107.46967824)(73.27623709,107.43301497)(72.98611905,107.35968842)
\curveto(72.69600102,107.28636188)(72.41704136,107.17637206)(72.1492401,107.02971897)
\lineto(72.1492401,107.84268719)
\curveto(72.47123924,107.96702351)(72.78367405,108.05947872)(73.08654453,108.12005282)
\curveto(73.38941501,108.18381503)(73.68431521,108.21569614)(73.97124514,108.21569614)
\curveto(74.74595594,108.21569614)(75.32459796,108.01484517)(75.7071712,107.61314322)
\curveto(76.08974444,107.21144128)(76.28103105,106.60251214)(76.28103105,105.7863558)
\closepath
}
}
{
\newrgbcolor{curcolor}{0 0 0}
\pscustom[linestyle=none,fillstyle=solid,fillcolor=curcolor]
{
\newpath
\moveto(38.70340132,141.60687238)
\curveto(38.70340132,142.25405885)(38.56950069,142.76096844)(38.30169942,143.12760117)
\curveto(38.03708627,143.49742201)(37.67204764,143.68233243)(37.20658353,143.68233243)
\curveto(36.74111943,143.68233243)(36.37448674,143.49742201)(36.10668548,143.12760117)
\curveto(35.84207232,142.76096844)(35.70976575,142.25405885)(35.70976575,141.60687238)
\curveto(35.70976575,140.95968591)(35.84207232,140.45118225)(36.10668548,140.08136141)
\curveto(36.37448674,139.71472869)(36.74111943,139.53141232)(37.20658353,139.53141232)
\curveto(37.67204764,139.53141232)(38.03708627,139.71472869)(38.30169942,140.08136141)
\curveto(38.56950069,140.45118225)(38.70340132,140.95968591)(38.70340132,141.60687238)
\closepath
\moveto(35.70976575,143.47669929)
\curveto(35.89467614,143.79551036)(36.1274082,144.03143055)(36.4079619,144.18445987)
\curveto(36.69170372,144.34067729)(37.02964341,144.418786)(37.42178098,144.418786)
\curveto(38.07215548,144.418786)(38.59978774,144.16054904)(39.00467775,143.6440751)
\curveto(39.41275586,143.12760117)(39.61679492,142.4485336)(39.61679492,141.60687238)
\curveto(39.61679492,140.76521116)(39.41275586,140.08614358)(39.00467775,139.56966965)
\curveto(38.59978774,139.05319572)(38.07215548,138.79495875)(37.42178098,138.79495875)
\curveto(37.02964341,138.79495875)(36.69170372,138.87147341)(36.4079619,139.02450272)
\curveto(36.1274082,139.18072015)(35.89467614,139.41823439)(35.70976575,139.73704546)
\lineto(35.70976575,138.93364157)
\lineto(34.82506514,138.93364157)
\lineto(34.82506514,146.37469191)
\lineto(35.70976575,146.37469191)
\lineto(35.70976575,143.47669929)
\closepath
}
}
{
\newrgbcolor{curcolor}{0 0 0}
\pscustom[linestyle=none,fillstyle=solid,fillcolor=curcolor]
{
\newpath
\moveto(147.99876688,141.04832664)
\lineto(147.99876688,140.22579408)
\curveto(147.75009427,140.36288284)(147.49982761,140.46490239)(147.2479669,140.53185271)
\curveto(146.9992943,140.60199115)(146.74743358,140.63706036)(146.49238476,140.63706036)
\curveto(145.92171301,140.63706036)(145.47856568,140.45533805)(145.16294276,140.09189344)
\curveto(144.84731984,139.73163693)(144.68950838,139.22472733)(144.68950838,138.57116464)
\curveto(144.68950838,137.91760195)(144.84731984,137.40909829)(145.16294276,137.04565368)
\curveto(145.47856568,136.68539717)(145.92171301,136.50526892)(146.49238476,136.50526892)
\curveto(146.74743358,136.50526892)(146.9992943,136.53874408)(147.2479669,136.6056944)
\curveto(147.49982761,136.67583284)(147.75009427,136.77944643)(147.99876688,136.91653519)
\lineto(147.99876688,136.10356697)
\curveto(147.75328238,135.98879498)(147.49823356,135.902716)(147.2336204,135.84533)
\curveto(146.97219536,135.78794401)(146.69323571,135.75925102)(146.39674145,135.75925102)
\curveto(145.59014954,135.75925102)(144.94933937,136.01270581)(144.47431093,136.51961541)
\curveto(143.9992825,137.02652501)(143.76176828,137.71037475)(143.76176828,138.57116464)
\curveto(143.76176828,139.44470697)(144.00087655,140.13174482)(144.4790931,140.6322782)
\curveto(144.96049776,141.13281157)(145.61884253,141.38307826)(146.45412743,141.38307826)
\curveto(146.72511681,141.38307826)(146.98972996,141.35438527)(147.2479669,141.29699927)
\curveto(147.50620383,141.24280139)(147.75647049,141.15991052)(147.99876688,141.04832664)
\closepath
}
}
{
\newrgbcolor{curcolor}{0 0 0}
\pscustom[linestyle=none,fillstyle=solid,fillcolor=curcolor]
{
\newpath
\moveto(200.16884401,81.51236054)
\lineto(200.16884401,84.41035315)
\lineto(201.04876245,84.41035315)
\lineto(201.04876245,76.96930281)
\lineto(200.16884401,76.96930281)
\lineto(200.16884401,77.77270671)
\curveto(199.98393361,77.45389564)(199.74960751,77.21638139)(199.46586569,77.06016397)
\curveto(199.18531198,76.90713466)(198.84737229,76.83062)(198.45204661,76.83062)
\curveto(197.80486022,76.83062)(197.27722796,77.08885696)(196.86914985,77.6053309)
\curveto(196.46425984,78.12180483)(196.26181483,78.8008724)(196.26181483,79.64253362)
\curveto(196.26181483,80.48419484)(196.46425984,81.16326242)(196.86914985,81.67973635)
\curveto(197.27722796,82.19621028)(197.80486022,82.45444725)(198.45204661,82.45444725)
\curveto(198.84737229,82.45444725)(199.18531198,82.37633854)(199.46586569,82.22012111)
\curveto(199.74960751,82.0670918)(199.98393361,81.83117161)(200.16884401,81.51236054)
\closepath
\moveto(197.17042627,79.64253362)
\curveto(197.17042627,78.99534715)(197.30273285,78.4868435)(197.567346,78.11702266)
\curveto(197.83514727,77.75038993)(198.20177995,77.56707357)(198.66724406,77.56707357)
\curveto(199.13270816,77.56707357)(199.49934085,77.75038993)(199.76714211,78.11702266)
\curveto(200.03494338,78.4868435)(200.16884401,78.99534715)(200.16884401,79.64253362)
\curveto(200.16884401,80.28972009)(200.03494338,80.79662969)(199.76714211,81.16326242)
\curveto(199.49934085,81.53308326)(199.13270816,81.71799368)(198.66724406,81.71799368)
\curveto(198.20177995,81.71799368)(197.83514727,81.53308326)(197.567346,81.16326242)
\curveto(197.30273285,80.79662969)(197.17042627,80.28972009)(197.17042627,79.64253362)
\closepath
}
}
{
\newrgbcolor{curcolor}{0 0 0}
\pscustom[linestyle=none,fillstyle=solid,fillcolor=curcolor]
{
\newpath
\moveto(96.93991196,16.65295578)
\lineto(96.93991196,16.22256084)
\lineto(92.89419999,16.22256084)
\curveto(92.93245731,15.61681981)(93.1141796,15.15454376)(93.43936685,14.83573269)
\curveto(93.76774221,14.52010973)(94.22364198,14.36229825)(94.80706617,14.36229825)
\curveto(95.14500586,14.36229825)(95.47178717,14.40374369)(95.78741009,14.48663457)
\curveto(96.10622112,14.56952545)(96.42184404,14.69386176)(96.73427885,14.85964352)
\lineto(96.73427885,14.02754663)
\curveto(96.41865593,13.89364598)(96.09506273,13.79162644)(95.76349926,13.721488)
\curveto(95.43193579,13.65134957)(95.09559015,13.61628035)(94.75446235,13.61628035)
\curveto(93.90004879,13.61628035)(93.22257535,13.86495298)(92.72204203,14.36229825)
\curveto(92.22469682,14.85964352)(91.97602422,15.53233487)(91.97602422,16.38037231)
\curveto(91.97602422,17.25710275)(92.21194438,17.95211088)(92.68378471,18.4653967)
\curveto(93.15881314,18.98187063)(93.79802926,19.2401076)(94.60143305,19.2401076)
\curveto(95.32194598,19.2401076)(95.89102367,19.00737552)(96.30866612,18.54191136)
\curveto(96.72949668,18.07963531)(96.93991196,17.44998345)(96.93991196,16.65295578)
\closepath
\moveto(96.05999352,16.91119274)
\curveto(96.0536173,17.39259746)(95.91812261,17.77676479)(95.65350945,18.06369476)
\curveto(95.39208441,18.35062472)(95.04458039,18.4940897)(94.61099739,18.4940897)
\curveto(94.1200284,18.4940897)(93.72629678,18.35540688)(93.42980252,18.07804125)
\curveto(93.13649637,17.80067562)(92.96752652,17.41013207)(92.92289298,16.90641058)
\lineto(96.05999352,16.91119274)
\closepath
}
}
{
\newrgbcolor{curcolor}{0 0 0}
\pscustom[linestyle=none,fillstyle=solid,fillcolor=curcolor]
{
\newpath
\moveto(16.85573754,79.41034958)
\lineto(16.85573754,78.67867817)
\lineto(16.01407642,78.67867817)
\curveto(15.6984535,78.67867817)(15.47847389,78.61491596)(15.35413759,78.48739153)
\curveto(15.2329894,78.35986711)(15.1724153,78.13032314)(15.1724153,77.79875962)
\lineto(15.1724153,77.32532519)
\lineto(16.62141143,77.32532519)
\lineto(16.62141143,76.64147545)
\lineto(15.1724153,76.64147545)
\lineto(15.1724153,71.96929924)
\lineto(14.28771469,71.96929924)
\lineto(14.28771469,76.64147545)
\lineto(13.44605357,76.64147545)
\lineto(13.44605357,77.32532519)
\lineto(14.28771469,77.32532519)
\lineto(14.28771469,77.69833414)
\curveto(14.28771469,78.29451084)(14.42639749,78.72809389)(14.70376308,78.9990833)
\curveto(14.98112868,79.27326082)(15.4210879,79.41034958)(16.02364075,79.41034958)
\lineto(16.85573754,79.41034958)
\closepath
}
}
\end{pspicture}

  \end{center}
  \caption{\label{fig:window-edge-merge}
    The visibility window (dashed line) passing through polygon vertex
    $a$ merges with the line segment $\overline{ab}$; i.e. the segment
    $\overline{ab}$ goes from being invisible to visible. Because of our
    general position and small motion assumptions, no two windows can
    change at the same time.
  }
\end{figure}

\begin{figure}
  \begin{center}
    %LaTeX with PSTricks extensions
%%Creator: inkscape 0.91
%%Please note this file requires PSTricks extensions
\psset{xunit=1.0pt,yunit=1.0pt,runit=.5pt}
\begin{pspicture}(216.14173228,177.16535433)
{
\newrgbcolor{curcolor}{0 0 0}
\pscustom[linewidth=0.83060648,linecolor=curcolor,linestyle=dashed,dash=2.49181795 2.49181795]
{
\newpath
\moveto(73.00270751,73.40022722)
\lineto(96.94585178,99.01164249)
\lineto(166.36328916,165.60251218)
}
}
{
\newrgbcolor{curcolor}{0 0 0}
\pscustom[linewidth=0.85157064,linecolor=curcolor,linestyle=dashed,dash=2.55471039 2.55471039]
{
\newpath
\moveto(97.33240201,69.31902479)
\lineto(96.19594134,103.226745)
\lineto(94.93334058,144.14564939)
}
}
{
\newrgbcolor{curcolor}{0 0 0}
\pscustom[linewidth=1.0000006,linecolor=curcolor]
{
\newpath
\moveto(16.4405738,79.10053062)
\lineto(96.19594134,99.24424263)
\lineto(59.32537936,136.6199149)
\lineto(106.66152358,146.11484056)
\lineto(95.02630064,164.85268173)
\lineto(184.56472001,165.61752219)
\lineto(116.90406968,10.35089964)
\closepath
}
}
{
\newrgbcolor{curcolor}{0.60000002 0.60000002 0.60000002}
\pscustom[linestyle=none,fillstyle=solid,fillcolor=curcolor]
{
\newpath
\moveto(22.94821438,79.45760499)
\curveto(22.94821438,76.66403873)(20.69139932,74.39506594)(17.89736284,74.37954391)
\curveto(15.10332635,74.36402189)(12.82143199,76.60778027)(12.79038271,79.40117404)
\curveto(12.75933343,82.19456781)(14.9907905,84.48847481)(17.78448194,84.53503898)
\curveto(20.57817338,84.58160314)(22.88486525,82.36333624)(22.94695997,79.57045993)
\lineto(17.86914174,79.45760499)
\closepath
}
}
{
\newrgbcolor{curcolor}{0.15686275 0.04313726 0.04313726}
\pscustom[linewidth=0.12953624,linecolor=curcolor,linestyle=dashed,dash=0.38860851 0.38860851]
{
\newpath
\moveto(22.94821438,79.45760499)
\curveto(22.94821438,76.66403873)(20.69139932,74.39506594)(17.89736284,74.37954391)
\curveto(15.10332635,74.36402189)(12.82143199,76.60778027)(12.79038271,79.40117404)
\curveto(12.75933343,82.19456781)(14.9907905,84.48847481)(17.78448194,84.53503898)
\curveto(20.57817338,84.58160314)(22.88486525,82.36333624)(22.94695997,79.57045993)
\lineto(17.86914174,79.45760499)
\closepath
}
}
{
\newrgbcolor{curcolor}{0.60000002 0.60000002 0.60000002}
\pscustom[linestyle=none,fillstyle=solid,fillcolor=curcolor]
{
\newpath
\moveto(101.27499942,99.24417685)
\curveto(101.27499942,96.45061059)(99.01818436,94.18163779)(96.22414788,94.16611576)
\curveto(93.4301114,94.15059374)(91.14821703,96.39435213)(91.11716775,99.18774589)
\curveto(91.08611847,101.98113966)(93.31757554,104.27504666)(96.11126698,104.32161083)
\curveto(98.90495842,104.36817499)(101.21165029,102.14990809)(101.27374501,99.35703178)
\lineto(96.19592678,99.24417685)
\closepath
}
}
{
\newrgbcolor{curcolor}{0.15686275 0.04313726 0.04313726}
\pscustom[linewidth=0.12953624,linecolor=curcolor,linestyle=dashed,dash=0.38860851 0.38860851]
{
\newpath
\moveto(101.27499942,99.24417685)
\curveto(101.27499942,96.45061059)(99.01818436,94.18163779)(96.22414788,94.16611576)
\curveto(93.4301114,94.15059374)(91.14821703,96.39435213)(91.11716775,99.18774589)
\curveto(91.08611847,101.98113966)(93.31757554,104.27504666)(96.11126698,104.32161083)
\curveto(98.90495842,104.36817499)(101.21165029,102.14990809)(101.27374501,99.35703178)
\lineto(96.19592678,99.24417685)
\closepath
}
}
{
\newrgbcolor{curcolor}{0.60000002 0.60000002 0.60000002}
\pscustom[linestyle=none,fillstyle=solid,fillcolor=curcolor]
{
\newpath
\moveto(64.40446353,136.61987051)
\curveto(64.40446353,133.82630425)(62.14764847,131.55733146)(59.35361199,131.54180943)
\curveto(56.55957551,131.5262874)(54.27768114,133.77004579)(54.24663186,136.56343956)
\curveto(54.21558258,139.35683333)(56.44703965,141.65074033)(59.24073109,141.69730449)
\curveto(62.03442253,141.74386866)(64.3411144,139.52560176)(64.40320912,136.73272544)
\lineto(59.32539089,136.61987051)
\closepath
}
}
{
\newrgbcolor{curcolor}{0.15686275 0.04313726 0.04313726}
\pscustom[linewidth=0.12953624,linecolor=curcolor,linestyle=dashed,dash=0.38860851 0.38860851]
{
\newpath
\moveto(64.40446353,136.61987051)
\curveto(64.40446353,133.82630425)(62.14764847,131.55733146)(59.35361199,131.54180943)
\curveto(56.55957551,131.5262874)(54.27768114,133.77004579)(54.24663186,136.56343956)
\curveto(54.21558258,139.35683333)(56.44703965,141.65074033)(59.24073109,141.69730449)
\curveto(62.03442253,141.74386866)(64.3411144,139.52560176)(64.40320912,136.73272544)
\lineto(59.32539089,136.61987051)
\closepath
}
}
{
\newrgbcolor{curcolor}{0.60000002 0.60000002 0.60000002}
\pscustom[linestyle=none,fillstyle=solid,fillcolor=curcolor]
{
\newpath
\moveto(111.19221878,146.76782578)
\curveto(111.19221878,143.97425952)(108.93540371,141.70528672)(106.14136723,141.6897647)
\curveto(103.34733075,141.67424267)(101.06543638,143.91800106)(101.0343871,146.71139483)
\curveto(101.00333783,149.5047886)(103.23479489,151.7986956)(106.02848633,151.84525976)
\curveto(108.82217777,151.89182392)(111.12886964,149.67355703)(111.19096436,146.88068071)
\lineto(106.11314613,146.76782578)
\closepath
}
}
{
\newrgbcolor{curcolor}{0.15686275 0.04313726 0.04313726}
\pscustom[linewidth=0.12953624,linecolor=curcolor,linestyle=dashed,dash=0.38860851 0.38860851]
{
\newpath
\moveto(111.19221878,146.76782578)
\curveto(111.19221878,143.97425952)(108.93540371,141.70528672)(106.14136723,141.6897647)
\curveto(103.34733075,141.67424267)(101.06543638,143.91800106)(101.0343871,146.71139483)
\curveto(101.00333783,149.5047886)(103.23479489,151.7986956)(106.02848633,151.84525976)
\curveto(108.82217777,151.89182392)(111.12886964,149.67355703)(111.19096436,146.88068071)
\lineto(106.11314613,146.76782578)
\closepath
}
}
{
\newrgbcolor{curcolor}{0.60000002 0.60000002 0.60000002}
\pscustom[linestyle=none,fillstyle=solid,fillcolor=curcolor]
{
\newpath
\moveto(100.10538203,164.13838057)
\curveto(100.10538203,161.3448143)(97.84856696,159.07584151)(95.05453048,159.06031948)
\curveto(92.260494,159.04479746)(89.97859964,161.28855585)(89.94755036,164.08194961)
\curveto(89.91650108,166.87534338)(92.14795814,169.16925038)(94.94164958,169.21581455)
\curveto(97.73534102,169.26237871)(100.04203289,167.04411181)(100.10412762,164.2512355)
\lineto(95.02630939,164.13838057)
\closepath
}
}
{
\newrgbcolor{curcolor}{0.15686275 0.04313726 0.04313726}
\pscustom[linewidth=0.12953624,linecolor=curcolor,linestyle=dashed,dash=0.38860851 0.38860851]
{
\newpath
\moveto(100.10538203,164.13838057)
\curveto(100.10538203,161.3448143)(97.84856696,159.07584151)(95.05453048,159.06031948)
\curveto(92.260494,159.04479746)(89.97859964,161.28855585)(89.94755036,164.08194961)
\curveto(89.91650108,166.87534338)(92.14795814,169.16925038)(94.94164958,169.21581455)
\curveto(97.73534102,169.26237871)(100.04203289,167.04411181)(100.10412762,164.2512355)
\lineto(95.02630939,164.13838057)
\closepath
}
}
{
\newrgbcolor{curcolor}{0.60000002 0.60000002 0.60000002}
\pscustom[linestyle=none,fillstyle=solid,fillcolor=curcolor]
{
\newpath
\moveto(121.98314165,10.35093539)
\curveto(121.98314165,7.55736912)(119.72632659,5.28839633)(116.9322901,5.2728743)
\curveto(114.13825362,5.25735228)(111.85635926,7.50111067)(111.82530998,10.29450443)
\curveto(111.7942607,13.0878982)(114.02571777,15.3818052)(116.81940921,15.42836937)
\curveto(119.61310065,15.47493353)(121.91979252,13.25666663)(121.98188724,10.46379032)
\lineto(116.90406901,10.35093539)
\closepath
}
}
{
\newrgbcolor{curcolor}{0.15686275 0.04313726 0.04313726}
\pscustom[linewidth=0.12953624,linecolor=curcolor,linestyle=dashed,dash=0.38860851 0.38860851]
{
\newpath
\moveto(121.98314165,10.35093539)
\curveto(121.98314165,7.55736912)(119.72632659,5.28839633)(116.9322901,5.2728743)
\curveto(114.13825362,5.25735228)(111.85635926,7.50111067)(111.82530998,10.29450443)
\curveto(111.7942607,13.0878982)(114.02571777,15.3818052)(116.81940921,15.42836937)
\curveto(119.61310065,15.47493353)(121.91979252,13.25666663)(121.98188724,10.46379032)
\lineto(116.90406901,10.35093539)
\closepath
}
}
{
\newrgbcolor{curcolor}{0 0 0}
\pscustom[linestyle=none,fillstyle=solid,fillcolor=curcolor]
{
\newpath
\moveto(99.79325687,69.31892708)
\curveto(99.79325687,67.96541991)(98.69981038,66.86608285)(97.34607544,66.85856229)
\curveto(95.9923405,66.85104174)(94.88674286,67.9381622)(94.87169921,69.29158579)
\curveto(94.85665557,70.64500938)(95.93781589,71.75642728)(97.29138366,71.77898802)
\curveto(98.64495142,71.80154877)(99.76256367,70.72677915)(99.7926491,69.37360627)
\lineto(97.33240207,69.31892708)
\closepath
}
}
{
\newrgbcolor{curcolor}{0.15686275 0.04313726 0.04313726}
\pscustom[linewidth=0.06276144,linecolor=curcolor,linestyle=dashed,dash=0.18828419 0.18828419]
{
\newpath
\moveto(99.79325687,69.31892708)
\curveto(99.79325687,67.96541991)(98.69981038,66.86608285)(97.34607544,66.85856229)
\curveto(95.9923405,66.85104174)(94.88674286,67.9381622)(94.87169921,69.29158579)
\curveto(94.85665557,70.64500938)(95.93781589,71.75642728)(97.29138366,71.77898802)
\curveto(98.64495142,71.80154877)(99.76256367,70.72677915)(99.7926491,69.37360627)
\lineto(97.33240207,69.31892708)
\closepath
}
}
{
\newrgbcolor{curcolor}{0 0 0}
\pscustom[linestyle=none,fillstyle=solid,fillcolor=curcolor]
{
\newpath
\moveto(93.63208981,70.11167526)
\lineto(77.23701004,73.21132711)
}
}
{
\newrgbcolor{curcolor}{0 0 0}
\pscustom[linewidth=0.51499999,linecolor=curcolor]
{
\newpath
\moveto(93.63208981,70.11167526)
\lineto(77.23701004,73.21132711)
}
}
{
\newrgbcolor{curcolor}{0 0 0}
\pscustom[linestyle=none,fillstyle=solid,fillcolor=curcolor]
{
\newpath
\moveto(82.22225401,69.94322597)
\lineto(76.56330019,73.3294664)
\lineto(83.0678894,74.41606832)
\curveto(81.84664533,73.27937928)(81.51070488,71.47180905)(82.22225401,69.94322597)
\closepath
}
}
{
\newrgbcolor{curcolor}{0 0 0}
\pscustom[linewidth=0.3540625,linecolor=curcolor]
{
\newpath
\moveto(82.22225401,69.94322597)
\lineto(76.56330019,73.3294664)
\lineto(83.0678894,74.41606832)
\curveto(81.84664533,73.27937928)(81.51070488,71.47180905)(82.22225401,69.94322597)
\closepath
}
}
{
\newrgbcolor{curcolor}{0 0 0}
\pscustom[linestyle=none,fillstyle=solid,fillcolor=curcolor]
{
\newpath
\moveto(76.38339185,73.37080937)
\curveto(76.38339185,72.01730221)(75.28994535,70.91796515)(73.93621041,70.91044459)
\curveto(72.58247547,70.90292403)(71.47687783,71.9900445)(71.46183419,73.34346809)
\curveto(71.44679054,74.69689168)(72.52795087,75.80830958)(73.88151863,75.83087032)
\curveto(75.2350864,75.85343106)(76.35269864,74.77866145)(76.38278407,73.42548857)
\lineto(73.92253704,73.37080937)
\closepath
}
}
{
\newrgbcolor{curcolor}{0.15686275 0.04313726 0.04313726}
\pscustom[linewidth=0.06276144,linecolor=curcolor,linestyle=dashed,dash=0.18828419 0.18828419]
{
\newpath
\moveto(76.38339185,73.37080937)
\curveto(76.38339185,72.01730221)(75.28994535,70.91796515)(73.93621041,70.91044459)
\curveto(72.58247547,70.90292403)(71.47687783,71.9900445)(71.46183419,73.34346809)
\curveto(71.44679054,74.69689168)(72.52795087,75.80830958)(73.88151863,75.83087032)
\curveto(75.2350864,75.85343106)(76.35269864,74.77866145)(76.38278407,73.42548857)
\lineto(73.92253704,73.37080937)
\closepath
}
}
{
\newrgbcolor{curcolor}{0 0 0}
\pscustom[linestyle=none,fillstyle=solid,fillcolor=curcolor]
{
\newpath
\moveto(96.67981339,99.52363173)
\curveto(95.96886479,99.52363173)(95.47630175,99.44233492)(95.20212426,99.27974129)
\curveto(94.92794678,99.11714766)(94.79085803,98.83978207)(94.79085803,98.4476445)
\curveto(94.79085803,98.13520969)(94.89287756,97.88653709)(95.09691662,97.70162669)
\curveto(95.30414379,97.5199044)(95.5846975,97.42904326)(95.93857774,97.42904326)
\curveto(96.42635862,97.42904326)(96.81690213,97.60120121)(97.11020828,97.94551713)
\curveto(97.40670254,98.29302115)(97.55494967,98.75370309)(97.55494967,99.32756294)
\lineto(97.55494967,99.52363173)
\lineto(96.67981339,99.52363173)
\closepath
\moveto(98.43486811,99.8870763)
\lineto(98.43486811,96.83127258)
\lineto(97.55494967,96.83127258)
\lineto(97.55494967,97.6442407)
\curveto(97.35409872,97.31905345)(97.10383206,97.07835112)(96.80414969,96.92213372)
\curveto(96.50446732,96.76910442)(96.13783464,96.69258978)(95.70425163,96.69258978)
\curveto(95.15589666,96.69258978)(94.71912555,96.84561907)(94.3939383,97.15167766)
\curveto(94.07193916,97.46092436)(93.91093959,97.87378465)(93.91093959,98.39025851)
\curveto(93.91093959,98.99281136)(94.11179054,99.44711708)(94.51349244,99.75317567)
\curveto(94.91838244,100.05923426)(95.52093529,100.21226355)(96.32115098,100.21226355)
\lineto(97.55494967,100.21226355)
\lineto(97.55494967,100.29834253)
\curveto(97.55494967,100.70323254)(97.42104903,101.01566735)(97.15324777,101.23564696)
\curveto(96.88863461,101.45881468)(96.51562571,101.57039854)(96.03422105,101.57039854)
\curveto(95.72816246,101.57039854)(95.43007415,101.53373528)(95.13995611,101.46040874)
\curveto(94.84983807,101.3870822)(94.57087842,101.2770924)(94.30307716,101.13043932)
\lineto(94.30307716,101.94340745)
\curveto(94.6250763,102.06774375)(94.93751111,102.16019895)(95.24038159,102.22077305)
\curveto(95.54325206,102.28453525)(95.83815227,102.31641636)(96.12508219,102.31641636)
\curveto(96.899793,102.31641636)(97.47843502,102.11556541)(97.86100826,101.71386351)
\curveto(98.24358149,101.31216161)(98.43486811,100.70323254)(98.43486811,99.8870763)
\closepath
}
}
{
\newrgbcolor{curcolor}{0 0 0}
\pscustom[linestyle=none,fillstyle=solid,fillcolor=curcolor]
{
\newpath
\moveto(60.85728797,135.70758861)
\curveto(60.85728797,136.354775)(60.72338734,136.86168454)(60.45558607,137.22831723)
\curveto(60.19097292,137.59813802)(59.82593429,137.78304842)(59.36047018,137.78304842)
\curveto(58.89500608,137.78304842)(58.52837339,137.59813802)(58.26057213,137.22831723)
\curveto(57.99595897,136.86168454)(57.86365239,136.354775)(57.86365239,135.70758861)
\curveto(57.86365239,135.06040222)(57.99595897,134.55189862)(58.26057213,134.18207783)
\curveto(58.52837339,133.81544514)(58.89500608,133.6321288)(59.36047018,133.6321288)
\curveto(59.82593429,133.6321288)(60.19097292,133.81544514)(60.45558607,134.18207783)
\curveto(60.72338734,134.55189862)(60.85728797,135.06040222)(60.85728797,135.70758861)
\closepath
\moveto(57.86365239,137.5774153)
\curveto(58.04856279,137.89622633)(58.28129484,138.1321465)(58.56184855,138.28517579)
\curveto(58.84559037,138.4413932)(59.18353006,138.5195019)(59.57566763,138.5195019)
\curveto(60.22604213,138.5195019)(60.75367439,138.26126496)(61.15856439,137.74479109)
\curveto(61.56664251,137.22831723)(61.77068157,136.54924973)(61.77068157,135.70758861)
\curveto(61.77068157,134.86592749)(61.56664251,134.18685999)(61.15856439,133.67038613)
\curveto(60.75367439,133.15391226)(60.22604213,132.89567532)(59.57566763,132.89567532)
\curveto(59.18353006,132.89567532)(58.84559037,132.97218997)(58.56184855,133.12521926)
\curveto(58.28129484,133.28143667)(58.04856279,133.51895089)(57.86365239,133.83776192)
\lineto(57.86365239,133.03435812)
\lineto(56.97895178,133.03435812)
\lineto(56.97895178,140.47540757)
\lineto(57.86365239,140.47540757)
\lineto(57.86365239,137.5774153)
\closepath
}
}
{
\newrgbcolor{curcolor}{0 0 0}
\pscustom[linestyle=none,fillstyle=solid,fillcolor=curcolor]
{
\newpath
\moveto(108.2324773,149.02374604)
\lineto(108.2324773,148.20121358)
\curveto(107.9838047,148.33830232)(107.73353804,148.44032185)(107.48167733,148.50727217)
\curveto(107.23300472,148.57741059)(106.98114401,148.61247981)(106.72609519,148.61247981)
\curveto(106.15542344,148.61247981)(105.71227611,148.43075752)(105.39665319,148.06731294)
\curveto(105.08103027,147.70705648)(104.92321881,147.20014694)(104.92321881,146.54658433)
\curveto(104.92321881,145.89302172)(105.08103027,145.38451812)(105.39665319,145.02107355)
\curveto(105.71227611,144.66081708)(106.15542344,144.48068885)(106.72609519,144.48068885)
\curveto(106.98114401,144.48068885)(107.23300472,144.51416401)(107.48167733,144.58111433)
\curveto(107.73353804,144.65125275)(107.9838047,144.75486634)(108.2324773,144.89195508)
\lineto(108.2324773,144.07898695)
\curveto(107.98699281,143.96421498)(107.73194399,143.878136)(107.46733083,143.82075002)
\curveto(107.20590579,143.76336403)(106.92694613,143.73467104)(106.63045188,143.73467104)
\curveto(105.82385997,143.73467104)(105.1830498,143.98812581)(104.70802136,144.49503535)
\curveto(104.23299293,145.00194489)(103.99547871,145.68579455)(103.99547871,146.54658433)
\curveto(103.99547871,147.42012655)(104.23458698,148.10716432)(104.71280353,148.60769764)
\curveto(105.19420818,149.10823096)(105.85255296,149.35849762)(106.68783786,149.35849762)
\curveto(106.95882724,149.35849762)(107.22344039,149.32980462)(107.48167733,149.27241864)
\curveto(107.73991426,149.21822076)(107.99018092,149.1353299)(108.2324773,149.02374604)
\closepath
}
}
{
\newrgbcolor{curcolor}{0 0 0}
\pscustom[linestyle=none,fillstyle=solid,fillcolor=curcolor]
{
\newpath
\moveto(96.6083117,165.25595963)
\lineto(96.6083117,168.1539519)
\lineto(97.48823015,168.1539519)
\lineto(97.48823015,160.71290245)
\lineto(96.6083117,160.71290245)
\lineto(96.6083117,161.51630625)
\curveto(96.42340131,161.19749522)(96.1890752,160.959981)(95.90533338,160.80376359)
\curveto(95.62477967,160.6507343)(95.28683998,160.57421965)(94.8915143,160.57421965)
\curveto(94.24432791,160.57421965)(93.71669566,160.83245659)(93.30861754,161.34893045)
\curveto(92.90372753,161.86540432)(92.70128252,162.54447182)(92.70128252,163.38613294)
\curveto(92.70128252,164.22779406)(92.90372753,164.90686155)(93.30861754,165.42333542)
\curveto(93.71669566,165.93980929)(94.24432791,166.19804623)(94.8915143,166.19804623)
\curveto(95.28683998,166.19804623)(95.62477967,166.11993753)(95.90533338,165.96372012)
\curveto(96.1890752,165.81069083)(96.42340131,165.57477066)(96.6083117,165.25595963)
\closepath
\moveto(93.60989396,163.38613294)
\curveto(93.60989396,162.73894655)(93.74220054,162.23044295)(94.00681369,161.86062216)
\curveto(94.27461496,161.49398947)(94.64124764,161.31067313)(95.10671175,161.31067313)
\curveto(95.57217585,161.31067313)(95.93880854,161.49398947)(96.2066098,161.86062216)
\curveto(96.47441107,162.23044295)(96.6083117,162.73894655)(96.6083117,163.38613294)
\curveto(96.6083117,164.03331933)(96.47441107,164.54022887)(96.2066098,164.90686155)
\curveto(95.93880854,165.27668235)(95.57217585,165.46159275)(95.10671175,165.46159275)
\curveto(94.64124764,165.46159275)(94.27461496,165.27668235)(94.00681369,164.90686155)
\curveto(93.74220054,164.54022887)(93.60989396,164.03331933)(93.60989396,163.38613294)
\closepath
}
}
{
\newrgbcolor{curcolor}{0 0 0}
\pscustom[linestyle=none,fillstyle=solid,fillcolor=curcolor]
{
\newpath
\moveto(118.30042291,14.16832545)
\lineto(118.30042291,13.43665413)
\lineto(117.45876179,13.43665413)
\curveto(117.14313887,13.43665413)(116.92315926,13.37289193)(116.79882295,13.24536752)
\curveto(116.67767476,13.1178431)(116.61710067,12.88829916)(116.61710067,12.55673569)
\lineto(116.61710067,12.08330131)
\lineto(118.0660968,12.08330131)
\lineto(118.0660968,11.39945165)
\lineto(116.61710067,11.39945165)
\lineto(116.61710067,6.727276)
\lineto(115.73240006,6.727276)
\lineto(115.73240006,11.39945165)
\lineto(114.89073894,11.39945165)
\lineto(114.89073894,12.08330131)
\lineto(115.73240006,12.08330131)
\lineto(115.73240006,12.45631022)
\curveto(115.73240006,13.05248684)(115.87108285,13.48606984)(116.14844845,13.75705922)
\curveto(116.42581405,14.03123671)(116.86577327,14.16832545)(117.46832612,14.16832545)
\lineto(118.30042291,14.16832545)
\closepath
}
}
{
\newrgbcolor{curcolor}{0.60000002 0.60000002 0.60000002}
\pscustom[linestyle=none,fillstyle=solid,fillcolor=curcolor]
{
\newpath
\moveto(190.06691484,165.36335688)
\curveto(190.06691484,162.56979062)(187.81009977,160.30081783)(185.01606329,160.2852958)
\curveto(182.22202681,160.26977377)(179.94013244,162.51353216)(179.90908317,165.30692593)
\curveto(179.87803389,168.1003197)(182.10949095,170.3942267)(184.90318239,170.44079086)
\curveto(187.69687383,170.48735503)(190.0035657,168.26908813)(190.06566043,165.47621181)
\lineto(184.98784219,165.36335688)
\closepath
}
}
{
\newrgbcolor{curcolor}{0.15686275 0.04313726 0.04313726}
\pscustom[linewidth=0.12953624,linecolor=curcolor,linestyle=dashed,dash=0.38860851 0.38860851]
{
\newpath
\moveto(190.06691484,165.36335688)
\curveto(190.06691484,162.56979062)(187.81009977,160.30081783)(185.01606329,160.2852958)
\curveto(182.22202681,160.26977377)(179.94013244,162.51353216)(179.90908317,165.30692593)
\curveto(179.87803389,168.1003197)(182.10949095,170.3942267)(184.90318239,170.44079086)
\curveto(187.69687383,170.48735503)(190.0035657,168.26908813)(190.06566043,165.47621181)
\lineto(184.98784219,165.36335688)
\closepath
}
}
{
\newrgbcolor{curcolor}{0 0 0}
\pscustom[linestyle=none,fillstyle=solid,fillcolor=curcolor]
{
\newpath
\moveto(187.53155684,165.30712375)
\lineto(187.53155684,164.87672886)
\lineto(183.48584486,164.87672886)
\curveto(183.52410219,164.27098791)(183.70582448,163.80871191)(184.03101173,163.48990088)
\curveto(184.35938709,163.17427796)(184.81528686,163.0164665)(185.39871105,163.0164665)
\curveto(185.73665074,163.0164665)(186.06343205,163.05791193)(186.37905497,163.1408028)
\curveto(186.697866,163.22369367)(187.01348892,163.34802997)(187.32592373,163.51381171)
\lineto(187.32592373,162.68171492)
\curveto(187.01030081,162.54781429)(186.68670761,162.44579476)(186.35514414,162.37565633)
\curveto(186.02358067,162.3055179)(185.68723503,162.27044869)(185.34610723,162.27044869)
\curveto(184.49169367,162.27044869)(183.81422023,162.51912129)(183.31368691,163.0164665)
\curveto(182.8163417,163.51381171)(182.5676691,164.18650298)(182.5676691,165.03454032)
\curveto(182.5676691,165.91127066)(182.80358926,166.6062787)(183.27542958,167.11956446)
\curveto(183.75045802,167.63603833)(184.38967414,167.89427527)(185.19307793,167.89427527)
\curveto(185.91359086,167.89427527)(186.48266855,167.66154321)(186.900311,167.19607911)
\curveto(187.32114156,166.73380312)(187.53155684,166.10415133)(187.53155684,165.30712375)
\closepath
\moveto(186.6516384,165.56536069)
\curveto(186.64526218,166.04676535)(186.50976749,166.43093264)(186.24515433,166.71786256)
\curveto(185.98372929,167.00479249)(185.63622526,167.14825746)(185.20264226,167.14825746)
\curveto(184.71167328,167.14825746)(184.31794165,167.00957466)(184.0214474,166.73220906)
\curveto(183.72814125,166.45484346)(183.5591714,166.06429995)(183.51453786,165.56057852)
\lineto(186.6516384,165.56536069)
\closepath
}
}
{
\newrgbcolor{curcolor}{0 0 0}
\pscustom[linestyle=none,fillstyle=solid,fillcolor=curcolor]
{
\newpath
\moveto(19.06279715,80.28781152)
\curveto(19.06279715,80.92543358)(18.93049057,81.41959068)(18.66587742,81.77028281)
\curveto(18.40445237,82.12097494)(18.03622563,82.29632101)(17.5611972,82.29632101)
\curveto(17.08935687,82.29632101)(16.72113013,82.12097494)(16.45651698,81.77028281)
\curveto(16.19509193,81.41959068)(16.06437941,80.92543358)(16.06437941,80.28781152)
\curveto(16.06437941,79.65337757)(16.19509193,79.16081452)(16.45651698,78.81012239)
\curveto(16.72113013,78.45943026)(17.08935687,78.28408419)(17.5611972,78.28408419)
\curveto(18.03622563,78.28408419)(18.40445237,78.45943026)(18.66587742,78.81012239)
\curveto(18.93049057,79.16081452)(19.06279715,79.65337757)(19.06279715,80.28781152)
\closepath
\moveto(19.9427156,78.21235171)
\curveto(19.9427156,77.30055216)(19.74027059,76.62307872)(19.33538058,76.17993139)
\curveto(18.93049057,75.73359595)(18.31040312,75.51042823)(17.47511822,75.51042823)
\curveto(17.16587152,75.51042823)(16.87415943,75.53433905)(16.59998194,75.58216071)
\curveto(16.32580446,75.62679425)(16.05959724,75.69693268)(15.80136031,75.79257599)
\lineto(15.80136031,76.6485836)
\curveto(16.05959724,76.50830675)(16.31464607,76.40469317)(16.56650678,76.33774285)
\curveto(16.8183675,76.27079253)(17.07501038,76.23731738)(17.33643542,76.23731738)
\curveto(17.91348339,76.23731738)(18.34547233,76.38875262)(18.63240226,76.69162309)
\curveto(18.91933219,76.99130546)(19.06279715,77.44561118)(19.06279715,78.05454025)
\lineto(19.06279715,78.48971731)
\curveto(18.88107486,78.17409439)(18.64834281,77.93817422)(18.36460099,77.78195682)
\curveto(18.08085918,77.62573941)(17.74132543,77.54763071)(17.34599975,77.54763071)
\curveto(16.68924903,77.54763071)(16.16002272,77.79789737)(15.75832082,78.29843069)
\curveto(15.35661892,78.79896401)(15.15576797,79.46209095)(15.15576797,80.28781152)
\curveto(15.15576797,81.1167202)(15.35661892,81.7814412)(15.75832082,82.28197451)
\curveto(16.16002272,82.78250783)(16.68924903,83.03277449)(17.34599975,83.03277449)
\curveto(17.74132543,83.03277449)(18.08085918,82.95466579)(18.36460099,82.79844838)
\curveto(18.64834281,82.64223098)(18.88107486,82.40631082)(19.06279715,82.0906879)
\lineto(19.06279715,82.90365602)
\lineto(19.9427156,82.90365602)
\lineto(19.9427156,78.21235171)
\closepath
}
}
\end{pspicture}

  \end{center}
  \caption{\label{fig:window-vertex-pass}
    The window in this figure is initially passing through $a$ and
    landing on the polygonal edge $\overline{bc}$. When we then move
    the point a short distance to the left, the window passes through
    point $c$ now landing on edge $\overline{de}$.
  }
\end{figure}
  
\begin{figure}
  \begin{center}
    \input{figures/edge-change.tex}
  \end{center}
  \caption{\label{fig:edge-change}
    Example point movement where a window is created when edge
    $\overline{ag}$ goes invisible.
  }
\end{figure}

\subsection{Delaunay Justification} \label{sec:delaunay-justification}

We use DTs to ensure that we look at a constant number of edges
(neighboring DTs) for each visibility update. If we had
used the entire convex partitions of our polygon, we would be stuck
with an $O(\log n)$ average case as required by the visibility
complex of \cite{dynamic-visibility}. In the Section~\ref{sec:future}
we discuss possible theories for improving the time bound to be
$\theta(1)$ in all cases.

In order to get the DTs and build a meaningful graph of visibility
polygons from them, we have to first partition our polygon into
convex \emph{visibility regions} (Figure~\ref{fig:visibility-regions}).
Top-level code for this looks like:

\begin{samepage}
\begin{minted}{python}
def PrecomputeVisibility(polygon):
  delaunays = []
  visGraph  = VisibilityPartition(polygon)
  for convexPoly in visGraph:
    delaunays.append(DelaunayTriang(convexPoly))
  ds = LinkNeighbors(delaunays)
  ComputeVisibilities(polygon, ds)
  return ds
\end{minted}
\end{samepage}

\emph{VisibilityPartition} computes a list of convex polygons like the
ones shown in Figure~\ref{fig:visibility-regions}. These visibility
regions are then further partitioned into their Delaunay Triangulations
in the \emph{DelaunayTriang} method. Next we link neighboring
DTs in \emph{LinkNeighbors} forming a Delaunay graph. Finally
\emph{ComputeVisibilities} informs each of the nodes (DTs) in the
Delaunay graph how to compute the visibility polygon of a point inside
of it:

\begin{minted}{python}
def ComputeVisibilities(polygon, ds):
  for d in ds:
    ws = ComputeWindows(polygon, d)
    vs = VisibleEdges(polygon, d)
    d.visEdges = RadialSort(d, ws, vs)
    def GetVisPoly(position, d=d):
      visPoly = []
      for e in d.visEdges:
        if isWindow(e):
          AddWindow(visPoly, point, e)
        else: visPoly.append(e)
      return visPoly
    d.getVisPoly = GetVisPoly
\end{minted}

\emph{ComputeVisibilities} takes each DT in the DT-partitioning of
the polygon. For a DT, we precompute all the polygonal edges which
will be involved in a window as well as any edges fully visible from
inside the DT. We then radially sort these windows \& edges so that
when \emph{GetVisPoly} iterates over them the DTs will already be in
the correct order to form the visibility polygon (\emph{visPoly})
computed online. We then define the code \emph{GetVisPoly} for computing
a visibility polygon in $O(v)$ time ($v$ vertices in the visibility
polygon). Note that \emph{GetVisPoly} does not run until a point
requests its visibility polygon at runtime by calling
\py{point.currDelaunay.getVisPoly(point)} e.g. immediately after
\emph{UpdateVisibility} runs.

As just stated \emph{GetVisPoly} runs in $O(v)$ time, but the
actual visibility update (\emph{UpdateVisibility}) to know which
DT we are in took constant average time.

\begin{figure}
  \begin{center}
    %LaTeX with PSTricks extensions
%%Creator: inkscape 0.91
%%Please note this file requires PSTricks extensions
\psset{xunit=1.0pt,yunit=1.0pt,runit=.5pt}
\begin{pspicture}(216.14173228,177.16535433)
{
\newrgbcolor{curcolor}{0.16862746 0.63921571 0}
\pscustom[linestyle=none,fillstyle=solid,fillcolor=curcolor]
{
\newpath
\moveto(26.07144454,79.64280094)
\lineto(101.42863046,98.57138223)
\lineto(142.85722515,58.57135838)
\lineto(122.85721323,12.49990092)
\closepath
}
}
{
\newrgbcolor{curcolor}{0 0 0}
\pscustom[linewidth=0,linecolor=curcolor]
{
\newpath
\moveto(26.07144454,79.64280094)
\lineto(101.42863046,98.57138223)
\lineto(142.85722515,58.57135838)
\lineto(122.85721323,12.49990092)
\closepath
}
}
{
\newrgbcolor{curcolor}{0 0 0}
\pscustom[linewidth=1.0000006,linecolor=curcolor,linestyle=dashed,dash=2 2]
{
\newpath
\moveto(65.75400019,137.33426533)
\lineto(188.92868261,161.07141948)
}
}
{
\newrgbcolor{curcolor}{0 0 0}
\pscustom[linewidth=1.0000006,linecolor=curcolor,linestyle=dashed,dash=2 2]
{
\newpath
\moveto(65.75400019,137.33426533)
\lineto(143.57151558,59.64278902)
}
}
{
\newrgbcolor{curcolor}{0 0 0}
\pscustom[linewidth=1.0000006,linecolor=curcolor,linestyle=dashed,dash=2 2]
{
\newpath
\moveto(24.29772748,80.17191126)
\lineto(170.35724154,117.49997351)
}
}
{
\newrgbcolor{curcolor}{0 0 0}
\pscustom[linewidth=1.0000006,linecolor=curcolor,linestyle=dashed,dash=2 2]
{
\newpath
\moveto(112.92853731,147.03148111)
\lineto(128.69593671,122.10144625)
\lineto(154.53467211,82.27269251)
}
}
{
\newrgbcolor{curcolor}{0 0 0}
\pscustom[linewidth=1.0000006,linecolor=curcolor]
{
\newpath
\moveto(22.86914263,79.81478105)
\lineto(102.62451117,99.95859305)
\lineto(65.75394819,137.33421533)
\lineto(113.09009741,146.82912099)
\lineto(101.45487047,165.56700216)
\lineto(190.99329384,166.33185261)
\lineto(123.33264351,11.06510007)
\closepath
}
}
{
\newrgbcolor{curcolor}{0.60000002 0.60000002 0.60000002}
\pscustom[linestyle=none,fillstyle=solid,fillcolor=curcolor]
{
\newpath
\moveto(29.3767861,80.17196089)
\curveto(29.3767861,77.3783941)(27.11997103,75.10942088)(24.32593455,75.09389885)
\curveto(21.53189807,75.07837682)(19.25000371,77.32213563)(19.21895443,80.11552993)
\curveto(19.18790515,82.90892422)(21.41936221,85.20283165)(24.21305365,85.24939582)
\curveto(27.00674509,85.29596)(29.31343696,83.07769268)(29.37553169,80.28481584)
\lineto(24.29771346,80.17196089)
\closepath
}
}
{
\newrgbcolor{curcolor}{0.15686275 0.04313726 0.04313726}
\pscustom[linewidth=0.12953624,linecolor=curcolor,linestyle=dashed,dash=0.38860851 0.38860851]
{
\newpath
\moveto(29.3767861,80.17196089)
\curveto(29.3767861,77.3783941)(27.11997103,75.10942088)(24.32593455,75.09389885)
\curveto(21.53189807,75.07837682)(19.25000371,77.32213563)(19.21895443,80.11552993)
\curveto(19.18790515,82.90892422)(21.41936221,85.20283165)(24.21305365,85.24939582)
\curveto(27.00674509,85.29596)(29.31343696,83.07769268)(29.37553169,80.28481584)
\lineto(24.29771346,80.17196089)
\closepath
}
}
{
\newrgbcolor{curcolor}{0.60000002 0.60000002 0.60000002}
\pscustom[linestyle=none,fillstyle=solid,fillcolor=curcolor]
{
\newpath
\moveto(107.70358449,99.95859378)
\curveto(107.70358449,97.16502699)(105.44676943,94.89605377)(102.65273295,94.88053174)
\curveto(99.85869647,94.86500971)(97.5768021,97.10876852)(97.54575282,99.90216281)
\curveto(97.51470354,102.69555711)(99.74616061,104.98946454)(102.53985205,105.03602871)
\curveto(105.33354349,105.08259288)(107.64023536,102.86432557)(107.70233008,100.07144873)
\lineto(102.62451185,99.95859378)
\closepath
}
}
{
\newrgbcolor{curcolor}{0.15686275 0.04313726 0.04313726}
\pscustom[linewidth=0.12953624,linecolor=curcolor,linestyle=dashed,dash=0.38860851 0.38860851]
{
\newpath
\moveto(107.70358449,99.95859378)
\curveto(107.70358449,97.16502699)(105.44676943,94.89605377)(102.65273295,94.88053174)
\curveto(99.85869647,94.86500971)(97.5768021,97.10876852)(97.54575282,99.90216281)
\curveto(97.51470354,102.69555711)(99.74616061,104.98946454)(102.53985205,105.03602871)
\curveto(105.33354349,105.08259288)(107.64023536,102.86432557)(107.70233008,100.07144873)
\lineto(102.62451185,99.95859378)
\closepath
}
}
{
\newrgbcolor{curcolor}{0.60000002 0.60000002 0.60000002}
\pscustom[linestyle=none,fillstyle=solid,fillcolor=curcolor]
{
\newpath
\moveto(70.83303334,137.33428744)
\curveto(70.83303334,134.54072066)(68.57621828,132.27174743)(65.7821818,132.2562254)
\curveto(62.98814532,132.24070338)(60.70625095,134.48446219)(60.67520167,137.27785648)
\curveto(60.64415239,140.07125077)(62.87560946,142.3651582)(65.6693009,142.41172238)
\curveto(68.46299234,142.45828655)(70.76968421,140.24001923)(70.83177893,137.44714239)
\lineto(65.7539607,137.33428744)
\closepath
}
}
{
\newrgbcolor{curcolor}{0.15686275 0.04313726 0.04313726}
\pscustom[linewidth=0.12953624,linecolor=curcolor,linestyle=dashed,dash=0.38860851 0.38860851]
{
\newpath
\moveto(70.83303334,137.33428744)
\curveto(70.83303334,134.54072066)(68.57621828,132.27174743)(65.7821818,132.2562254)
\curveto(62.98814532,132.24070338)(60.70625095,134.48446219)(60.67520167,137.27785648)
\curveto(60.64415239,140.07125077)(62.87560946,142.3651582)(65.6693009,142.41172238)
\curveto(68.46299234,142.45828655)(70.76968421,140.24001923)(70.83177893,137.44714239)
\lineto(65.7539607,137.33428744)
\closepath
}
}
{
\newrgbcolor{curcolor}{0.60000002 0.60000002 0.60000002}
\pscustom[linestyle=none,fillstyle=solid,fillcolor=curcolor]
{
\newpath
\moveto(117.62078096,147.48218167)
\curveto(117.62078096,144.68861489)(115.36396589,142.41964167)(112.56992941,142.40411964)
\curveto(109.77589293,142.38859761)(107.49399857,144.63235642)(107.46294929,147.42575071)
\curveto(107.43190001,150.219145)(109.66335707,152.51305243)(112.45704851,152.55961661)
\curveto(115.25073995,152.60618078)(117.55743182,150.38791347)(117.61952655,147.59503663)
\lineto(112.54170832,147.48218167)
\closepath
}
}
{
\newrgbcolor{curcolor}{0.15686275 0.04313726 0.04313726}
\pscustom[linewidth=0.12953624,linecolor=curcolor,linestyle=dashed,dash=0.38860851 0.38860851]
{
\newpath
\moveto(117.62078096,147.48218167)
\curveto(117.62078096,144.68861489)(115.36396589,142.41964167)(112.56992941,142.40411964)
\curveto(109.77589293,142.38859761)(107.49399857,144.63235642)(107.46294929,147.42575071)
\curveto(107.43190001,150.219145)(109.66335707,152.51305243)(112.45704851,152.55961661)
\curveto(115.25073995,152.60618078)(117.55743182,150.38791347)(117.61952655,147.59503663)
\lineto(112.54170832,147.48218167)
\closepath
}
}
{
\newrgbcolor{curcolor}{0.60000002 0.60000002 0.60000002}
\pscustom[linestyle=none,fillstyle=solid,fillcolor=curcolor]
{
\newpath
\moveto(106.53393658,164.85273646)
\curveto(106.53393658,162.05916967)(104.27712152,159.79019645)(101.48308503,159.77467442)
\curveto(98.68904855,159.75915239)(96.40715419,162.0029112)(96.37610491,164.7963055)
\curveto(96.34505563,167.58969979)(98.57651269,169.88360722)(101.37020414,169.93017139)
\curveto(104.16389558,169.97673557)(106.47058745,167.75846825)(106.53268217,164.96559141)
\lineto(101.45486394,164.85273646)
\closepath
}
}
{
\newrgbcolor{curcolor}{0.15686275 0.04313726 0.04313726}
\pscustom[linewidth=0.12953624,linecolor=curcolor,linestyle=dashed,dash=0.38860851 0.38860851]
{
\newpath
\moveto(106.53393658,164.85273646)
\curveto(106.53393658,162.05916967)(104.27712152,159.79019645)(101.48308503,159.77467442)
\curveto(98.68904855,159.75915239)(96.40715419,162.0029112)(96.37610491,164.7963055)
\curveto(96.34505563,167.58969979)(98.57651269,169.88360722)(101.37020414,169.93017139)
\curveto(104.16389558,169.97673557)(106.47058745,167.75846825)(106.53268217,164.96559141)
\lineto(101.45486394,164.85273646)
\closepath
}
}
{
\newrgbcolor{curcolor}{0.60000002 0.60000002 0.60000002}
\pscustom[linestyle=none,fillstyle=solid,fillcolor=curcolor]
{
\newpath
\moveto(128.41171909,11.06516921)
\curveto(128.41171909,8.27160242)(126.15490403,6.0026292)(123.36086755,5.98710717)
\curveto(120.56683106,5.97158514)(118.2849367,8.21534395)(118.25388742,11.00873825)
\curveto(118.22283814,13.80213254)(120.45429521,16.09603997)(123.24798665,16.14260414)
\curveto(126.04167809,16.18916832)(128.34836996,13.970901)(128.41046468,11.17802416)
\lineto(123.33264645,11.06516921)
\closepath
}
}
{
\newrgbcolor{curcolor}{0.15686275 0.04313726 0.04313726}
\pscustom[linewidth=0.12953624,linecolor=curcolor,linestyle=dashed,dash=0.38860851 0.38860851]
{
\newpath
\moveto(128.41171909,11.06516921)
\curveto(128.41171909,8.27160242)(126.15490403,6.0026292)(123.36086755,5.98710717)
\curveto(120.56683106,5.97158514)(118.2849367,8.21534395)(118.25388742,11.00873825)
\curveto(118.22283814,13.80213254)(120.45429521,16.09603997)(123.24798665,16.14260414)
\curveto(126.04167809,16.18916832)(128.34836996,13.970901)(128.41046468,11.17802416)
\lineto(123.33264645,11.06516921)
\closepath
}
}
{
\newrgbcolor{curcolor}{0 0 0}
\pscustom[linestyle=none,fillstyle=solid,fillcolor=curcolor]
{
\newpath
\moveto(103.10840609,100.23798762)
\curveto(102.39745749,100.23798762)(101.90489445,100.15669081)(101.63071696,99.99409718)
\curveto(101.35653947,99.83150356)(101.21945073,99.55413796)(101.21945073,99.16200039)
\curveto(101.21945073,98.84956559)(101.32147026,98.60089298)(101.52550932,98.41598258)
\curveto(101.73273649,98.2342603)(102.0132902,98.14339915)(102.36717044,98.14339915)
\curveto(102.85495132,98.14339915)(103.24549483,98.31555711)(103.53880098,98.65987302)
\curveto(103.83529524,99.00737705)(103.98354237,99.46805898)(103.98354237,100.04191884)
\lineto(103.98354237,100.23798762)
\lineto(103.10840609,100.23798762)
\closepath
\moveto(104.86346081,100.6014322)
\lineto(104.86346081,97.54562847)
\lineto(103.98354237,97.54562847)
\lineto(103.98354237,98.3585966)
\curveto(103.78269142,98.03340935)(103.53242476,97.79270702)(103.23274239,97.63648961)
\curveto(102.93306002,97.48346032)(102.56642733,97.40694567)(102.13284433,97.40694567)
\curveto(101.58448936,97.40694567)(101.14771825,97.55997497)(100.822531,97.86603356)
\curveto(100.50053186,98.17528026)(100.33953229,98.58814054)(100.33953229,99.10461441)
\curveto(100.33953229,99.70716726)(100.54038324,100.16147298)(100.94208513,100.46753156)
\curveto(101.34697514,100.77359015)(101.94952799,100.92661945)(102.74974368,100.92661945)
\lineto(103.98354237,100.92661945)
\lineto(103.98354237,101.01269843)
\curveto(103.98354237,101.41758844)(103.84964173,101.73002325)(103.58184047,101.95000286)
\curveto(103.31722731,102.17317058)(102.94421841,102.28475444)(102.46281375,102.28475444)
\curveto(102.15675516,102.28475444)(101.85866685,102.24809117)(101.56854881,102.17476463)
\curveto(101.27843077,102.1014381)(100.99947112,101.99144829)(100.73166985,101.84479522)
\lineto(100.73166985,102.65776334)
\curveto(101.053669,102.78209965)(101.36610381,102.87455484)(101.66897428,102.93512894)
\curveto(101.97184476,102.99889115)(102.26674497,103.03077225)(102.55367489,103.03077225)
\curveto(103.3283857,103.03077225)(103.90702772,102.8299213)(104.28960095,102.4282194)
\curveto(104.67217419,102.0265175)(104.86346081,101.41758844)(104.86346081,100.6014322)
\closepath
}
}
{
\newrgbcolor{curcolor}{0 0 0}
\pscustom[linestyle=none,fillstyle=solid,fillcolor=curcolor]
{
\newpath
\moveto(67.2858616,136.4219445)
\curveto(67.2858616,137.0691309)(67.15196096,137.57604043)(66.8841597,137.94267312)
\curveto(66.61954654,138.31249392)(66.25450791,138.49740431)(65.78904381,138.49740431)
\curveto(65.3235797,138.49740431)(64.95694702,138.31249392)(64.68914575,137.94267312)
\curveto(64.4245326,137.57604043)(64.29222602,137.0691309)(64.29222602,136.4219445)
\curveto(64.29222602,135.77475811)(64.4245326,135.26625452)(64.68914575,134.89643372)
\curveto(64.95694702,134.52980104)(65.3235797,134.3464847)(65.78904381,134.3464847)
\curveto(66.25450791,134.3464847)(66.61954654,134.52980104)(66.8841597,134.89643372)
\curveto(67.15196096,135.26625452)(67.2858616,135.77475811)(67.2858616,136.4219445)
\closepath
\moveto(64.29222602,138.2917712)
\curveto(64.47713642,138.61058223)(64.70986847,138.84650239)(64.99042218,138.99953169)
\curveto(65.27416399,139.15574909)(65.61210369,139.23385779)(66.00424125,139.23385779)
\curveto(66.65461576,139.23385779)(67.18224801,138.97562086)(67.58713802,138.45914699)
\curveto(67.99521614,137.94267312)(68.1992552,137.26360562)(68.1992552,136.4219445)
\curveto(68.1992552,135.58028338)(67.99521614,134.90121589)(67.58713802,134.38474202)
\curveto(67.18224801,133.86826815)(66.65461576,133.61003122)(66.00424125,133.61003122)
\curveto(65.61210369,133.61003122)(65.27416399,133.68654586)(64.99042218,133.83957516)
\curveto(64.70986847,133.99579256)(64.47713642,134.23330678)(64.29222602,134.55211781)
\lineto(64.29222602,133.74871401)
\lineto(63.40752541,133.74871401)
\lineto(63.40752541,141.18976347)
\lineto(64.29222602,141.18976347)
\lineto(64.29222602,138.2917712)
\closepath
}
}
{
\newrgbcolor{curcolor}{0 0 0}
\pscustom[linestyle=none,fillstyle=solid,fillcolor=curcolor]
{
\newpath
\moveto(114.66107,149.73810193)
\lineto(114.66107,148.91556947)
\curveto(114.4123974,149.05265821)(114.16213074,149.15467774)(113.91027003,149.22162806)
\curveto(113.66159742,149.29176649)(113.40973671,149.3268357)(113.15468788,149.3268357)
\curveto(112.58401614,149.3268357)(112.14086881,149.14511341)(111.82524589,148.78166884)
\curveto(111.50962297,148.42141237)(111.35181151,147.91450284)(111.35181151,147.26094022)
\curveto(111.35181151,146.60737761)(111.50962297,146.09887402)(111.82524589,145.73542944)
\curveto(112.14086881,145.37517298)(112.58401614,145.19504475)(113.15468788,145.19504475)
\curveto(113.40973671,145.19504475)(113.66159742,145.2285199)(113.91027003,145.29547022)
\curveto(114.16213074,145.36560865)(114.4123974,145.46922223)(114.66107,145.60631097)
\lineto(114.66107,144.79334285)
\curveto(114.41558551,144.67857088)(114.16053669,144.5924919)(113.89592353,144.53510591)
\curveto(113.63449849,144.47771993)(113.35553883,144.44902693)(113.05904458,144.44902693)
\curveto(112.25245267,144.44902693)(111.6116425,144.7024817)(111.13661406,145.20939124)
\curveto(110.66158563,145.71630078)(110.42407141,146.40015044)(110.42407141,147.26094022)
\curveto(110.42407141,148.13448245)(110.66317968,148.82152022)(111.14139623,149.32205353)
\curveto(111.62280088,149.82258685)(112.28114566,150.07285351)(113.11643056,150.07285351)
\curveto(113.38741994,150.07285351)(113.65203309,150.04416052)(113.91027003,149.98677453)
\curveto(114.16850696,149.93257666)(114.41877362,149.84968579)(114.66107,149.73810193)
\closepath
}
}
{
\newrgbcolor{curcolor}{0 0 0}
\pscustom[linestyle=none,fillstyle=solid,fillcolor=curcolor]
{
\newpath
\moveto(103.03688914,165.97031553)
\lineto(103.03688914,168.86830779)
\lineto(103.91680759,168.86830779)
\lineto(103.91680759,161.42725834)
\lineto(103.03688914,161.42725834)
\lineto(103.03688914,162.23066214)
\curveto(102.85197875,161.91185111)(102.61765264,161.67433689)(102.33391082,161.51811949)
\curveto(102.05335711,161.36509019)(101.71541742,161.28857555)(101.32009174,161.28857555)
\curveto(100.67290535,161.28857555)(100.1452731,161.54681248)(99.73719498,162.06328635)
\curveto(99.33230497,162.57976022)(99.12985996,163.25882771)(99.12985996,164.10048883)
\curveto(99.12985996,164.94214995)(99.33230497,165.62121745)(99.73719498,166.13769132)
\curveto(100.1452731,166.65416519)(100.67290535,166.91240212)(101.32009174,166.91240212)
\curveto(101.71541742,166.91240212)(102.05335711,166.83429342)(102.33391082,166.67807602)
\curveto(102.61765264,166.52504672)(102.85197875,166.28912656)(103.03688914,165.97031553)
\closepath
\moveto(100.0384714,164.10048883)
\curveto(100.0384714,163.45330244)(100.17077798,162.94479885)(100.43539113,162.57497805)
\curveto(100.7031924,162.20834537)(101.06982508,162.02502903)(101.53528919,162.02502903)
\curveto(102.00075329,162.02502903)(102.36738598,162.20834537)(102.63518724,162.57497805)
\curveto(102.90298851,162.94479885)(103.03688914,163.45330244)(103.03688914,164.10048883)
\curveto(103.03688914,164.74767523)(102.90298851,165.25458476)(102.63518724,165.62121745)
\curveto(102.36738598,165.99103824)(102.00075329,166.17594864)(101.53528919,166.17594864)
\curveto(101.06982508,166.17594864)(100.7031924,165.99103824)(100.43539113,165.62121745)
\curveto(100.17077798,165.25458476)(100.0384714,164.74767523)(100.0384714,164.10048883)
\closepath
}
}
{
\newrgbcolor{curcolor}{0 0 0}
\pscustom[linestyle=none,fillstyle=solid,fillcolor=curcolor]
{
\newpath
\moveto(124.72901561,14.88268134)
\lineto(124.72901561,14.15101003)
\lineto(123.88735449,14.15101003)
\curveto(123.57173157,14.15101003)(123.35175195,14.08724782)(123.22741565,13.95972341)
\curveto(123.10626746,13.832199)(123.04569337,13.60265506)(123.04569337,13.27109159)
\lineto(123.04569337,12.79765721)
\lineto(124.4946895,12.79765721)
\lineto(124.4946895,12.11380754)
\lineto(123.04569337,12.11380754)
\lineto(123.04569337,7.44163189)
\lineto(122.16099276,7.44163189)
\lineto(122.16099276,12.11380754)
\lineto(121.31933164,12.11380754)
\lineto(121.31933164,12.79765721)
\lineto(122.16099276,12.79765721)
\lineto(122.16099276,13.17066611)
\curveto(122.16099276,13.76684274)(122.29967555,14.20042574)(122.57704115,14.47141512)
\curveto(122.85440675,14.7455926)(123.29436597,14.88268134)(123.89691882,14.88268134)
\lineto(124.72901561,14.88268134)
\closepath
}
}
{
\newrgbcolor{curcolor}{0.60000002 0.60000002 0.60000002}
\pscustom[linestyle=none,fillstyle=solid,fillcolor=curcolor]
{
\newpath
\moveto(196.49549228,166.07771278)
\curveto(196.49549228,163.28414599)(194.23867721,161.01517277)(191.44464073,160.99965074)
\curveto(188.65060425,160.98412871)(186.36870988,163.22788752)(186.33766061,166.02128181)
\curveto(186.30661133,168.81467611)(188.53806839,171.10858354)(191.33175983,171.15514771)
\curveto(194.12545127,171.20171188)(196.43214314,168.98344457)(196.49423787,166.19056773)
\lineto(191.41641964,166.07771278)
\closepath
}
}
{
\newrgbcolor{curcolor}{0.15686275 0.04313726 0.04313726}
\pscustom[linewidth=0.12953624,linecolor=curcolor,linestyle=dashed,dash=0.38860851 0.38860851]
{
\newpath
\moveto(196.49549228,166.07771278)
\curveto(196.49549228,163.28414599)(194.23867721,161.01517277)(191.44464073,160.99965074)
\curveto(188.65060425,160.98412871)(186.36870988,163.22788752)(186.33766061,166.02128181)
\curveto(186.30661133,168.81467611)(188.53806839,171.10858354)(191.33175983,171.15514771)
\curveto(194.12545127,171.20171188)(196.43214314,168.98344457)(196.49423787,166.19056773)
\lineto(191.41641964,166.07771278)
\closepath
}
}
{
\newrgbcolor{curcolor}{0 0 0}
\pscustom[linestyle=none,fillstyle=solid,fillcolor=curcolor]
{
\newpath
\moveto(193.96014954,166.02147965)
\lineto(193.96014954,165.59108476)
\lineto(189.91443756,165.59108476)
\curveto(189.95269489,164.9853438)(190.13441717,164.52306781)(190.45960443,164.20425678)
\curveto(190.78797979,163.88863386)(191.24387956,163.7308224)(191.82730375,163.7308224)
\curveto(192.16524344,163.7308224)(192.49202474,163.77226783)(192.80764766,163.8551587)
\curveto(193.1264587,163.93804956)(193.44208162,164.06238587)(193.75451643,164.2281676)
\lineto(193.75451643,163.39607081)
\curveto(193.43889351,163.26217018)(193.11530031,163.16015065)(192.78373684,163.09001222)
\curveto(192.45217337,163.0198738)(192.11582773,162.98480458)(191.77469993,162.98480458)
\curveto(190.92028636,162.98480458)(190.24281292,163.23347719)(189.74227961,163.7308224)
\curveto(189.2449344,164.2281676)(188.9962618,164.90085888)(188.9962618,165.74889622)
\curveto(188.9962618,166.62562655)(189.23218196,167.3206346)(189.70402228,167.83392036)
\curveto(190.17905072,168.35039423)(190.81826683,168.60863116)(191.62167063,168.60863116)
\curveto(192.34218356,168.60863116)(192.91126125,168.37589911)(193.3289037,167.910435)
\curveto(193.74973426,167.44815901)(193.96014954,166.81850723)(193.96014954,166.02147965)
\closepath
\moveto(193.0802311,166.27971658)
\curveto(193.07385488,166.76112124)(192.93836019,167.14528853)(192.67374703,167.43221846)
\curveto(192.41232199,167.71914839)(192.06481796,167.86261335)(191.63123496,167.86261335)
\curveto(191.14026598,167.86261335)(190.74653435,167.72393055)(190.45004009,167.44656496)
\curveto(190.15673395,167.16919936)(189.9877641,166.77865585)(189.94313056,166.27493442)
\lineto(193.0802311,166.27971658)
\closepath
}
}
{
\newrgbcolor{curcolor}{0 0 0}
\pscustom[linestyle=none,fillstyle=solid,fillcolor=curcolor]
{
\newpath
\moveto(25.49137173,81.00210638)
\curveto(25.49137173,81.63972844)(25.35906515,82.13388554)(25.094452,82.48457767)
\curveto(24.83302695,82.8352698)(24.46480021,83.01061587)(23.98977178,83.01061587)
\curveto(23.51793145,83.01061587)(23.14970471,82.8352698)(22.88509156,82.48457767)
\curveto(22.62366651,82.13388554)(22.49295399,81.63972844)(22.49295399,81.00210638)
\curveto(22.49295399,80.36767243)(22.62366651,79.87510938)(22.88509156,79.52441725)
\curveto(23.14970471,79.17372512)(23.51793145,78.99837905)(23.98977178,78.99837905)
\curveto(24.46480021,78.99837905)(24.83302695,79.17372512)(25.094452,79.52441725)
\curveto(25.35906515,79.87510938)(25.49137173,80.36767243)(25.49137173,81.00210638)
\closepath
\moveto(26.37129017,78.92664657)
\curveto(26.37129017,78.01484702)(26.16884517,77.33737358)(25.76395516,76.89422625)
\curveto(25.35906515,76.44789081)(24.7389777,76.22472309)(23.9036928,76.22472309)
\curveto(23.5944461,76.22472309)(23.30273401,76.24863391)(23.02855652,76.29645557)
\curveto(22.75437903,76.34108911)(22.48817182,76.41122754)(22.22993489,76.50687085)
\lineto(22.22993489,77.36287846)
\curveto(22.48817182,77.22260161)(22.74322065,77.11898803)(22.99508136,77.05203771)
\curveto(23.24694208,76.98508739)(23.50358496,76.95161223)(23.76501,76.95161223)
\curveto(24.34205797,76.95161223)(24.77404691,77.10304747)(25.06097684,77.40591795)
\curveto(25.34790677,77.70560032)(25.49137173,78.15990604)(25.49137173,78.76883511)
\lineto(25.49137173,79.20401217)
\curveto(25.30964944,78.88838925)(25.07691739,78.65246908)(24.79317557,78.49625168)
\curveto(24.50943376,78.34003427)(24.16990001,78.26192557)(23.77457433,78.26192557)
\curveto(23.11782361,78.26192557)(22.5885973,78.51219223)(22.1868954,79.01272555)
\curveto(21.7851935,79.51325886)(21.58434255,80.17638581)(21.58434255,81.00210638)
\curveto(21.58434255,81.83101506)(21.7851935,82.49573605)(22.1868954,82.99626937)
\curveto(22.5885973,83.49680269)(23.11782361,83.74706935)(23.77457433,83.74706935)
\curveto(24.16990001,83.74706935)(24.50943376,83.66896065)(24.79317557,83.51274324)
\curveto(25.07691739,83.35652584)(25.30964944,83.12060567)(25.49137173,82.80498275)
\lineto(25.49137173,83.61795088)
\lineto(26.37129017,83.61795088)
\lineto(26.37129017,78.92664657)
\closepath
}
}
\end{pspicture}

  \end{center}
  \caption{\label{fig:visibility-regions}
    Example decomposition of a polygon into convex
    \emph{visibility regions}. Highlighted (in green) is one such
    region. Transitioning from one region to another neighboring
    region changes at most the visibility of one edge or which edge
    a window lands on.
  }
\end{figure}

\begin{figure}
  \begin{center}
    \input{figures/delaunay-triangles.tex}
  \end{center}
  \caption{\label{fig:delaunay-triangles}
    Further decomposition of the visibility regions of
    Figure~\ref{fig:visibility-regions} into their respective
    Delaunay Triangulations. One such DT is highlighted in red
    which has two neighboring DTs. Note that a point movement from
    the red to yellow DT is possible, but this can be handled in
    constant time as two transitions using the purple DT as an
    intermediary.
  }
\end{figure}



\section{Future} \label{sec:future}

The core limitation to updating a visibility polygon of $O(v)$ size in
$O(1)$ time is that to actually observe the solution necessarily
requires $O(w)$ time to compute the window point intersections where
in the worst case $w \in O(v)$. Certainly then if the solution is
required frequently relative to the distance the visibility travels,
any dynamic visibility algorithm has a tight lower bound of
$\Omega(w)$.

\subsection{Practical Improvements} \label{sec:improvements}

Possibilities for making the \emph{UpdateVisibility} algorithm $\theta(1)$:
\begin{itemize}
\item Find the $k$ highest degree vertices in our precomputed Delaunay
graph. Since we do not actually care about the Delaunay properties, we
could find a way to retriangulate the high degree vertices.
\item Come up with a triangulation which enforces a constant number of
neighbors of triangulation vertices. This has the downside of possibly
creating `long and skinny' triangles which the particle can easily
jump through if the velocity does not have a low enough upper bound.
\item Rely on online techniques to optimize visibility transitions
commonly taken at the cost of space. In the worst case though this
begins to require at least $\Omega(n!)$ space to remember
all possible transitions.
\end{itemize}

The \emph{GetVisPoly} algorithm presented here runs in $O(v)$ time for
$v$ output vertices in the visibility polygon. In practice the completely
visible edge remain unmodified so a more sophisticated algorithm could
run in $O(w)$ for $w$ window vertices in the output visibility polygon.

\subsection{Example Use Case} \label{sec:use-cases}

We have implemented (in a github repository - see~\cite{visibility-github})
a simple app allowing a user to control the motion of a point through a
polygon.

As the basis for an \emph{art-gallery-guard} game, this app would
require low-latency in recomputing the visibility polygon as a guard
moves around the polygon. Our \emph{UpdateVisibility} algorithm becomes
even more applicable when the game becomes multiplayer and the app needs
to compute a visibility polygon for each player.
As the basis for a multi-particle simulation the same logic applies -
we trade off memory usage in favor of constant-time low latency solutions.



%\appendix

%\acks
%Acknowledgments, if needed.

% We recommend abbrvnat bibliography style.
\bibliographystyle{abbrvnat}

% The bibliography should be embedded for final submission.

\begin{thebibliography}{}
\softraggedright

%\bibitem[Smith et~al.(2009)Smith, Jones]{smith02}
%P. Q. Smith, and X. Y. Jones. ...reference text...

% bibtex:
%@MISC{Rivière97dynamicvisibility,
%    author = {Stephane Rivière},
%    title = {Dynamic Visibility in Polygonal Scenes With the Visibility Complex},
%    year = {1997}
%}

\bibitem[Rivière (1997)]{dynamic-visibility}
Stepahne Rivière.
Dynamic Visibility in Polygonal Scenes With the Visibility Complex.
In \emph{SCG Proceedings of the thirteenth annual symposium on Computational geometry}.

\bibitem[Cronburg (2015)]{visibility-github}
Karl Cronburg
Visibility polygon implementation:
\href{https://github.com/cronburg/geo163}{https://github.com/cronburg/geo163}

\end{thebibliography}


\end{document}

%                       Revision History
%                       -------- -------
%  Date         Person  Ver.    Change
%  ----         ------  ----    ------

%  2013.06.29   TU      0.1--4  comments on permission/copyright notices


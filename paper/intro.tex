\section{Introduction} \label{sec:intro}

Given effectively unlimited precomputation time, and space we show
that the visibility of a point inside a simple polygon reduces to
(at runtime) the problem of detecting segment intersection with a
much smaller convex polygon.

The results of this work are directly applicable to the implementation
of games and simulations involving the continuous motion of a point
through a closed and finite partition of a $2$-dimensional space.
For example one might have a particle simulation involving at least
$p \in \Omega(n^2)$ particles ($n$ polygon vertices). In this situation,
the naiive worst case space complexity requires computing $O(n^2)$
(mostly) distinct visibility polygons.

Furthermore, many dynamic visibility applications realistically have
the self-imposed limitation of small point velocities. Relatively
low velocities lends itself to amortizing the cost of computing the
next segment intersection to cause a \emph{change} in the visibility
polygon. By \emph{change} we mean:

\begin{itemize}
\item A visibility \emph{window} merges with an edge in the polygon
(Figure~\ref{fig:window-edge-merge}).
\item A visibility \emph{window} passes over a vertex in the polygon,
changing which edge it lands on (Figure~\ref{fig:window-vertex-pass}).
\item A \emph{window} is otherwise created or removed / an edge becomes
visibile or invisible (Figure~\ref{fig:edge-change}).
\end{itemize}

All of these cases are defined by our point crossing over an infinite ray
extending from an edge on a reflex vertex, pointing into the polygon
(Figure \ref{fig:reflex-half-planes}). As we know from the algorithm
for computing the kernel of a polygon, these rays intersect with each
other to form a partitioning of the closed space inside our polygon
into convex subpolygons. In the \emph{Algorithm} section we discuss how
further decomposing these convex subpolygons into Delaunay triangulations
allows us to detect visibility changes in constant time.
% Number of partitions is $\omega(1)$ and $O(n^2)$ convex subpolygons.

